Ground-level ozone is a secondary air pollutant produced during the degradation of emitted volatile organic compounds (VOCs) and nitrogen oxides (\ce{NO_x}) in the presence of sunlight. 
As ozone production is dependent on photochemical processes, meteorological factors such as temperature influence ozone production.
Temperature directly influences ozone production through speeding up the rates of the chemical processes producing ozone and increasing the emissions of VOCs, such as isoprene, from vegetation.
In this study, we used a box model to reproduce the non-linear relationship of ozone on \ce{NO_x} and temperature from previous observational studes.
Faster chemistry was responsible for an increase in ozone of up to $20$~ppbv while increased isoprene emissions added a further $11$~ppbv of ozone under high-\ce{NO_x} conditions.
The shorter lifetime of peroxy nitrates with increased temperature was the main contributor to the increased production of ozone with temperature. 
At $40$~\degree C, the thermal decomposition of peroxy nitrates was responsible for up to $45$~\% of the normalised \ce{O_x} production.
The rate of increase in ozone with temperature from our box model simulations was about half rate of the increase in ozone with temperature over central Europe compared to both observed and WRF-Chem simulations.
The missing sensitivity in our simulations compared to observations and 3D model output is related to the indirect influence of temperature on ozone production not included in our experiment.
