Ground-level ozone is a secondary air pollutant produced during the degradation of emitted volatile organic compounds (VOC) and nitrogen oxides (\ce{NO_x}) in the presence of sunlight. 
As ozone production is dependent on photochemical processes, meteorological factors influence the ozone levels in many regions.
Temperature directly influences ozone production through speeding up the rates of the chemical processes producing ozone and increasing the emissions of VOCs, such as isoprene, from vegetation.
In this study, we used a box model to reproduce the non-linear relationship of ozone on \ce{NO_x} and temperature from previous observational studes.
Faster chemistry was responsible for an increase in ozone of up to $20$~ppbv while increased isoprene emissions added a further $11$~ppbv of ozone under high-\ce{NO_x} conditions.
The decrease in the lifetime of peroxy nitrates with increased temperature was the main contributor to the increased production of ozone with temperature, at $40$~\degree C the thermal decomposition of peroxy nitrates was responsible for up to $45$~\% of the normalised \ce{O_x} production.
The rate of increase in ozone with temperature from our box model simulations was up to half the rate of the observed increase in ozone with temperature over central Europe or the model output from the 3D WRF-Chem model.
The lack of sensitivity of the box model results to temperature was most likely due to only considering instantaneous ozone production thus omitting the influence of stagnant atmospheric conditions on ozone production.
