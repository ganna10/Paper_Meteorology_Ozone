The secondary air pollutant, ground-level ozone is produced during the degradation of emitted VOC and \ce{NO_x} in the presence of sunlight. 
As ozone production is dependent on photochemical processes, meteorological factors can influence the ozone levels in many regions.
For example, temperature is a major driver of ozone over central Europe during the summertime as well as other regions in the US.
Temperature mainly influences ozone production through speeding up the chemical processes producing ozone and increasing the emissions on important precursors from vegetation, one important example being isoprene emissions.
In this study, using a box model we determine that faster chemistry was responsible for an increase in ozone of up to 20~ppbv while increased isoprene emissions added a further increase of 10~ppbv of ozone.
The decrease in the lifetime of peroxy nitrates with increased temperature was the main contributor to the increased production of ozone with temperature, at $40$~\degree C the thermal decomposition of peroxy nitrates was responsible for up to $45$~\% of the normalised \ce{O_x} production.
The rate of increase in ozone with temperature from the box model simulations was half the rate of the observed increase in ozone with temperature over central Europe.
A similar lack of sensitivity of the box model simulations compared to model output from the 3-D WRF-Chem model also resulted due to the box model set-up only considering instantaneous ozone production and not simulating stagnant conditions.
