Surface ozone is a secondary air pollutant produced during the degradation of emitted volatile organic compounds (VOCs) in the presence of sunlight and nitrogen oxides (\ce{NO_x}). 
Temperature directly influences ozone production through speeding up the rates of the chemical reactions and increasing the emissions of VOCs, such as isoprene, from vegetation.
In this study, we used a box model to examine the non-linear relationship between ozone, \ce{NO_x}, and temperature and was compared to previous observational studies.
Under high-\ce{NO_x} conditions, an increase in ozone from $20$~\degree C to $40$~\degree C of up to $20$~ppbv was due to faster reaction rates while increased isoprene emissions added up to a further $11$~ppbv of ozone.
The increased oxidation rate of emitted VOC with temperature controlled the rate of \ce{O_x} production.
The rate of increase in ozone mixing ratios with temperature from our box model simulations was about half the rate of increase in ozone with temperature observed over central Europe or simulated by a regional chemistry model.
Modifying the box model setup to approximate stagnant conditions increased the rate of increase of ozone with temperature as the accumulation of oxidants enhanced ozone production through the increased production of peroxy radicals from the secondary degradation of emitted VOCs.
The model simulations approximating stagnant conditions and VOC-and-\ce{NO_x}-sensitive chemistry reproduced the $2$~ppbv increase in ozone per \degree C from the observational and model data over central Europe.
The simulated ozone-temperature relationship was more sensitive to mixing than the choice of chemical mechanism.
