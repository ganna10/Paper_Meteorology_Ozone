Ground-level ozone is a secondary air pollutant produced during the degradation of emitted volatile organic compounds (VOCs) in the presence of sunlight and nitrogen oxides (\ce{NO_x}). 
Many studies have shown that meteorological factors such as temperature influence ozone production.
Temperature directly influences ozone production through speeding up the rates of the chemical processes producing ozone and increasing the emissions of VOCs, such as isoprene, from vegetation.
In this study, we used a box model to reproduce the non-linear relationship of ozone on \ce{NO_x} and temperature from previous observational studes.
An increase in ozone of up to $20$~ppbv was due to faster reaction rates while increased isoprene emissions added a further $11$~ppbv of ozone under high-\ce{NO_x} conditions.
Increased VOC reactivity with temperature produced enhanced the net production of \ce{O_x} by $\sim1$~molecule of \ce{O_x} per loss of VOC.
The rate of increase in ozone with temperature from our box model simulations was about half the rate of increase in ozone with temperature over central Europe compared to both observed values and WRF-Chem output.
The missing sensitivity in our simulations compared to observations and 3D model output is related to not including stagnant atmospheric conditions (coupling of high temperature and low wind speeds) in our experiment.
