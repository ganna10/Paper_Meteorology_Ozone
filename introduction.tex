Surface-level ozone (\ce{O3}) is a secondary air pollutant formed during the photochemical degradation of volatile organic compounds (VOCs) in the presence of nitrogen oxides (\ce{NO_x} $\equiv$ NO + \ce{NO2}).
Due to the photochemical nature of ozone production, meteorological variables such as temperature strongly influence ozone production \citep{Jacob:2009}.
A study by \citet{Otero:2016} indicated that temperature is a major meteorological driver for ozone in many areas of central europe during the summertime.

Temperature primarily influences ozone production in two ways: speeding up the reaction rates of many chemical reactions leading to ozone production and increasing emissions of VOCs from biogenic sources (BVOCs).
In general, emissions of anthropogenic VOCs (AVOCs) are not typically temperature dependent, although evaporative emissions of AVOCs tend to increase with temperature \citep{Rubin:2006}.
The review of \citet{Pusede:2015} provides further details of the temperature-dependent processes impacting ozone production.

Many studies over the US \citep{Sillman:1995a, Dawson:2007, Pusede:2014} noted that increased temperatures tend to lead to higher ozone levels, often exceeding local air quality guidelines.
Some studies \citep{Sillman:1995a, Dawson:2007} included modelling experiments using regional chemical tranport models simulating the observed increases in ozone with temperature.
In these studies, the increase of ozone with temperature was attibuted to the decrease in the lifetime of PAN (peroxy acetyl nitrate) at higher temperatures and increased emissions of BVOCs, in particular isoprene, from vegetation.

\citet{Pusede:2014} used an analytical model constrained by observations over San Joaquin Valley, California to infer a non-linear relationship of ozone production with temperature and \ce{NO_x}, similar to the well-known non-linear relationship of ozone production on \ce{NO_x} and VOC levels \citep{Sillman:1999}.
Morever, \citet{Pusede:2014} showed that temperature can be used as a surrogate for VOC levels when considering the relationship of ozone across \ce{NO_x} gradients.

Environmental chamber studies have also been used to analyse the relationship of ozone with temperature for a particular mixture of VOCs.
The chamber experiments of \citet{Carter:1979} and \citet{Hatakeyama:1991}, also showed inreases in ozone with temperature linked to increased PAN decomposition at higher temperatures (T \textgreater 303~K).

Despite many studies considering the effects of temperature on ozone production from an observational and chamber study perspective, there have not been (to our knowledge) modelling studies focusing on the detailed chemical processes of the influence of temperature on ozone production.
Where regional modelling has accompanied observational studies, these have only considered the relationship of temperature on ozone under the same \ce{NO_x} conditions.
The review of \citet{Pusede:2015} also highlights a lack of modelling studies looking at the non-linear relationship of ozone on temperature across \ce{NO_x} gradients.

In this study, we use an idealised box model to determine how ozone levels vary with temperature and across \ce{NO_x} gradients.
We separate the effects of temperature-dependent chemistry and temperature-dependent BVOC emissions on ozone production by performing simulations using a temperature-independent source of isoprene followed by simulations using a temperature-dependent source of isoprene at differing temperatures and \ce{NO_x} emissions. 
