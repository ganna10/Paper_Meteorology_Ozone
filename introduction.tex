\textbf{Gap:}
Many observational studies have noted the dependence of ozone production on temperature and also the non-linearity of ozone production on temperature and NOx.
Furthermore, many regional modelling studies have also reproduced this relationship of ozone production on temperature for specific areas; currently most of these modelling studies have been concerned with US regions.
Despite all this research, there has not been (to our knowledge) a detailed process study looking at modelled ozone as a function of both NOx and temperature.
In this study, we model ozone over various temperature and NOx levels to determine how ozone production varies under these range of conditions.
The review of \citet{Pusede:2015} also highlights a lack of modelling studies looking at this non-linear relationship of ozone on NOx and temperature.

\citet{Pusede:2014} demonstrated the importance of tackling high ozone levels also from the stand point of temperature and in relation to varying NOx conditions over the San Joaquin Valley in California and the review of \citet{Pusede:2015} highlights that the temperature dependence of biogenic VOC (BVOC) emissions is mainly responsible for this relationship.
In this study, we model the effects of temperature-dependent and temperature-independent BVOC emissions on ozone production in order to look at the effects of temperature dependent ozone production chemistry without temperature dependent VOC emissions and then the effects of temperature-dependent chemistry and temperature-dependent BVOC emissions.
As mentioned in the review of \citet{Pusede:2015}, typically BVOC emissions are temperature-dependent while anthropogenic VOC (AVOC) emissions tend to be temperature-independent, as AVOC emissions tend to be process and combustion related.
