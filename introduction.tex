Surface-level ozone (\ce{O3}) is a secondary air pollutant formed during the photochemical degradation of volatile organic compounds (VOCs) in the presence of nitrogen oxides (\ce{NO_x} $\equiv$ NO + \ce{NO2}).
Due to the photochemical nature of ozone production, meteorological variables such as temperature strongly influence ozone production \citep{Jacob:2009}.
\citet{Otero:2016} showed that temperature was a major meteorological driver for summertime ozone in many areas of central Europe.

Temperature primarily influences ozone production in two ways: speeding up the reaction rates of many chemical reactions leading to ozone production and increasing emissions of VOCs from biogenic sources (BVOCs).
In general, emissions of anthropogenic VOCs (AVOCs) are not typically dependent on temperature, however evaporative emissions of AVOCs increase with temperature \citep{Rubin:2006}.
The review of \citet{Pusede:2015} provides further details of the temperature-dependent processes impacting ozone production.

Studies over the US \citep{Sillman:1995a, Dawson:2007, Pusede:2014} noted that increased temperatures tend to lead to higher ozone levels, often exceeding local air quality guidelines.
Some studies \citep{Sillman:1995a, Dawson:2007} included regional modelling to simulate the observed increases in ozone with temperature.
In these studies, the increase of ozone with temperature was attributed to the shorter lifetime of PAN (peroxy acetyl nitrate) at higher temperatures and increased emissions of BVOCs, in particular isoprene, from vegetation.

\citet{Pusede:2014} used an analytical model constrained by observations over San Joaquin Valley, California to infer a non-linear relationship of ozone production with temperature and \ce{NO_x}, similar to the well-known non-linear relationship of ozone production on \ce{NO_x} and VOC levels \citep{Sillman:1999}.
Morever, \citet{Pusede:2014} showed that temperature can be used as a surrogate for VOC levels when considering the relationship of ozone across \ce{NO_x} gradients.

Environmental chamber studies have also been used to analyse the relationship of ozone with temperature.
The chamber experiments of \citet{Carter:1979} and \citet{Hatakeyama:1991} showed increases in ozone from a VOC mix with temperature linked to increased PAN decomposition at temperatures greater than $303$~K.

Despite many studies considering the effects of temperature on ozone production from an observational and chamber study perspective, there are no modelling studies (to our knowledge) focusing on the detailed chemical processes of the influence of temperature on ozone production across \ce{NO_x} gradients.
Regional modelling studies have concentrated on reproducing ozone levels over regions with known meteorology and \ce{NO_x} conditions then only varying the temperature.
These modelling studies did not consider the relationship of ozone with \ce{NO_x} with temperature.
The review of \citet{Pusede:2015} also highlights a lack of modelling studies looking at the non-linear relationship of ozone on temperature across \ce{NO_x} gradients.

In this study, we use an idealised box model to determine how ozone levels vary with temperature across \ce{NO_x} gradients.
We determine whether faster chemistry or increased BVOC emissions have a greater influence on instantaneous ozone production with higher temperature at different \ce{NO_x} conditions.
\citet{Rasmussen:2013} indicated that changing the chemical mechanism used by a model may also change the simulated ozone-temperature relationship to investigate this, we repeated all simulations using various chemical mechanisms.
