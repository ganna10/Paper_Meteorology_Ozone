%%%%Supplement
{
    \begin{landscape}%
        \centering%
        %\tiny
\begin{longtable}{lllllllllllllll}
	\hline \hline
	\textbf{Type} & \textbf{MCM.species} & \textbf{SNAP.1} & \textbf{SNAP.2} & \textbf{SNAP.34} & \textbf{SNAP.5} & \textbf{SNAP.6} & \textbf{SNAP.71} & \textbf{SNAP.72} & \textbf{SNAP.73} & \textbf{SNAP.74} & \textbf{SNAP.8} & \textbf{SNAP.9} & \textbf{BVOC} & \textbf{Total}\\
	\endhead
	\hline
	Ethane & C2H6 & 4.15E+08 & 1.11E+09 & 2.98E+09 &  &  & 1.74E+08 & 4.62E+07 & 8.17E+06 &  & 8.30E+07 & 8.22E+07 &  & 4.91E+09 \\
	\hline Propane & C3H8 & 1.14E+09 & 4.72E+08 & 1.03E+08 & 3.12E+10 & 3.18E+08 & 8.49E+06 & 3.15E+07 & 8.17E+07 & 2.71E+06 & 7.53E+07 & 3.56E+07 &  & 3.35E+10 \\ \hline
	\multirow{2}{*}{Butanes} & NC4H10 & 7.77E+08 & 2.42E+08 & 1.27E+06 & 1.23E+11 & 1.18E+09 & 1.89E+08 & 3.26E+07 &  & 4.48E+07 & 1.40E+08 & 2.20E+07 &  & 1.25E+11 \\
	 & IC4H10 & 9.48E+07 & 8.49E+07 & 3.11E+05 & 2.98E+10 & 5.36E+07 & 8.81E+07 & 1.52E+07 &  & 2.09E+07 & 7.02E+07 & 2.20E+07 &  & 3.03E+10 \\
	\hline \multirow{3}{*}{Pentanes} & NC5H12 & 6.21E+08 & 2.25E+08 &  & 8.78E+10 &  & 1.13E+08 & 1.31E+07 &  & 2.25E+07 & 4.51E+07 & 1.11E+07 &  & 8.89E+10 \\
	 & IC5H12 & 2.62E+08 & 1.21E+08 &  & 5.25E+10 &  & 2.19E+08 & 2.54E+07 &  & 4.37E+07 & 8.60E+07 & 1.11E+07 &  & 5.33E+10 \\
	 & NEOP &  &  &  &  &  &  &  &  &  &  & 1.11E+07 &  & 1.11E+07 \\
	\hline \parbox[t]{2mm}{\multirow{14}{*}{\rotatebox[origin=c]{90}{Hexane and Higher Alkanes}}} & NC6H14 & 3.89E+08 & 2.39E+07 & 3.15E+08 & 1.26E+10 & 1.05E+09 & 3.98E+08 & 1.94E+08 &  & 8.35E+06 & 1.04E+08 & 3.84E+06 &  & 1.51E+10 \\
	 & M2PE &  &  & 4.06E+07 & 1.94E+09 & 2.20E+08 &  &  &  &  & 1.73E+08 & 1.65E+06 &  & 2.37E+09 \\
	 & M3PE &  &  & 3.04E+07 & 9.69E+08 & 2.20E+08 &  &  &  &  & 1.04E+08 &  &  & 1.32E+09 \\
	 & NC7H16 & 1.67E+08 & 4.11E+07 & 1.48E+08 & 1.35E+10 & 3.79E+08 & 6.55E+07 & 3.20E+07 &  & 1.38E+06 & 2.98E+07 & 1.94E+07 &  & 1.44E+10 \\
	 & M2HEX &  &  &  &  & 1.42E+08 & 5.10E+07 & 2.49E+07 &  & 1.07E+06 & 4.48E+07 &  &  & 2.64E+08 \\
	 & M3HEX &  &  &  &  & 1.42E+08 & 3.64E+07 & 1.78E+07 &  & 7.64E+05 & 2.98E+07 &  &  & 2.27E+08 \\
	 & M22C4 &  &  &  &  &  &  &  &  &  & 3.47E+07 &  &  & 3.47E+07 \\
	 & M23C4 &  &  &  &  &  &  &  &  &  & 3.47E+07 &  &  & 3.47E+07 \\
	 & NC8H18 &  &  & 6.13E+07 & 1.01E+10 & 4.16E+07 & 5.75E+07 & 2.81E+07 &  & 1.21E+06 & 1.70E+08 & 6.63E+06 &  & 1.04E+10 \\
	 & NC9H20 &  &  & 3.41E+07 &  & 1.00E+09 &  &  &  &  &  & 2.21E+06 &  & 1.04E+09 \\
	 & NC10H22 &  &  & 4.30E+07 &  & 1.94E+09 & 2.56E+07 & 1.25E+07 &  & 5.38E+05 &  & 3.32E+06 &  & 2.02E+09 \\
	 & NC11H24 &  &  & 1.68E+07 &  & 7.90E+08 & 9.33E+06 & 4.56E+06 &  & 1.96E+05 & 1.91E+07 & 1.21E+06 &  & 8.41E+08 \\
	 & NC12H26 &  &  &  &  & 5.58E+07 & 1.52E+08 & 7.44E+07 &  & 3.20E+06 & 1.76E+07 &  &  & 3.03E+08 \\
	 & CHEX &  & 3.81E+07 & 1.04E+07 &  & 2.26E+08 &  &  &  &  &  & 1.12E+06 &  & 2.75E+08 \\
	\hline Ethene & C2H4 & 8.93E+07 & 2.49E+09 & 3.11E+10 &  &  & 9.61E+08 & 5.94E+08 & 4.38E+07 &  & 1.18E+09 & 1.43E+08 &  & 3.66E+10 \\ \hline
	Propene & C3H6 & 5.95E+07 & 5.21E+08 & 5.33E+08 &  &  & 3.38E+08 & 9.90E+07 & 1.95E+07 &  & 2.06E+08 & 4.10E+07 &  & 1.82E+09 \\
	\hline \parbox[t]{2mm}{\multirow{11}{*}{\rotatebox[origin=c]{90}{Higher Alkenes}}} & HEX1ENE & 5.05E+06 & 1.28E+07 &  &  &  &  &  &  &  &  & 1.63E+07 &  & 3.42E+07 \\
	 & BUT1ENE &  & 1.80E+07 & 6.24E+07 &  &  &  &  &  &  & 1.96E+07 &  &  & 9.99E+07 \\
	 & MEPROPENE &  &  &  &  &  &  &  &  &  & 9.80E+06 &  &  & 9.80E+06 \\
	 & TBUT2ENE &  &  &  &  &  &  &  &  &  & 9.80E+06 &  &  & 9.80E+06 \\
	 & CBUT2ENE &  &  &  &  &  &  &  &  &  & 9.80E+06 &  &  & 9.80E+06 \\
	 & CPENT2ENE &  & 5.65E+06 &  &  &  &  &  &  &  & 3.92E+06 &  &  & 9.57E+06 \\
	 & TPENT2ENE &  & 5.65E+06 &  &  &  &  &  &  &  & 3.92E+06 &  &  & 9.57E+06 \\
	 & PENT1ENE &  & 5.14E+06 & 5.93E+06 &  &  &  &  &  &  & 1.57E+07 &  &  & 2.68E+07 \\
	 & ME2BUT2ENE &  & 3.08E+06 &  &  &  &  &  &  &  & 7.84E+06 &  &  & 1.09E+07 \\
	 & ME3BUT1ENE &  & 3.08E+06 &  &  &  &  &  &  &  & 7.84E+06 &  &  & 1.09E+07 \\
	 & ME2BUT1ENE &  & 2.05E+06 &  &  &  &  &  &  &  &  &  &  & 2.05E+06 \\
	\hline Ethyne & C2H2 & 6.97E+05 & 7.84E+08 & 3.45E+08 &  &  & 8.95E+08 & 2.80E+08 & 1.73E+07 & 1.09E+07 & 3.95E+08 & 5.38E+07 &  & 2.78E+09 \\ \hline
	Benzene & BENZENE & 6.91E+07 & 4.64E+08 & 5.74E+08 & 3.05E+09 &  & 2.16E+08 & 3.56E+07 &  & 1.53E+06 & 7.98E+07 & 2.75E+07 &  & 4.52E+09 \\
	\hline Toluene & TOLUENE & 8.49E+07 & 1.54E+08 & 4.87E+07 & 2.59E+09 & 2.16E+09 & 4.88E+08 & 2.26E+07 &  & 1.30E+06 & 6.79E+07 & 1.81E+07 &  & 5.63E+09 \\ \hline
	\multirow{3}{*}{Xylenes} & MXYL & 4.20E+07 & 1.32E+07 & 1.60E+06 & 3.74E+08 & 1.25E+09 & 1.04E+08 & 9.52E+06 &  & 2.05E+05 & 1.86E+07 & 3.66E+06 &  & 1.81E+09 \\
	 & OXYL & 9.33E+06 & 1.32E+07 & 6.42E+05 & 3.74E+08 & 3.12E+08 & 1.04E+08 & 9.52E+06 &  & 2.05E+05 & 1.51E+07 & 2.19E+06 &  & 8.40E+08 \\
	 & PXYL &  & 1.32E+07 & 6.42E+05 & 3.74E+08 & 3.12E+08 & 7.79E+07 & 7.14E+06 &  & 1.53E+05 & 1.86E+07 & 2.93E+06 &  & 8.07E+08 \\
	\hline \multirow{3}{*}{Trimethylbenzenes} & TM123B & 6.21E+03 & 1.06E+06 &  &  & 2.09E+07 & 1.79E+07 &  &  &  & 3.33E+06 & 3.30E+05 &  & 4.35E+07 \\
	 & TM124B & 6.21E+03 & 1.06E+06 & 1.46E+07 &  & 7.11E+07 & 7.50E+07 &  &  &  & 7.76E+06 & 4.40E+05 &  & 1.70E+08 \\
	 & TM135B & 6.21E+03 & 1.06E+06 &  &  & 2.09E+07 & 2.86E+07 &  &  &  & 3.33E+06 & 4.40E+05 &  & 5.43E+07 \\
	\hline \parbox[t]{2mm}{\multirow{11}{*}{\rotatebox[origin=c]{90}{Other Aromatics}}} & EBENZ & 1.36E+07 &  & 1.65E+07 &  & 5.68E+07 & 7.76E+07 & 5.32E+07 & 1.53E+04 &  & 1.74E+08 & 3.93E+06 &  & 3.96E+08 \\
	 & PBENZ &  &  &  &  & 1.26E+07 & 6.86E+07 & 4.70E+07 & 1.35E+04 &  & 2.79E+07 & 1.73E+06 &  & 1.58E+08 \\
	 & IPBENZ &  &  &  &  & 4.60E+07 &  &  &  &  & 2.79E+07 & 1.73E+06 &  & 7.57E+07 \\
	 & PETHTOL &  &  &  &  & 4.18E+06 &  &  &  &  & 5.59E+07 &  &  & 6.00E+07 \\
	 & METHTOL &  &  &  &  & 1.26E+07 &  &  &  &  & 5.59E+07 &  &  & 6.84E+07 \\
	 & OETHTOL &  &  &  &  &  &  &  &  &  & 4.19E+07 &  &  & 4.19E+07 \\
	 & DIET35TOL &  &  &  &  &  & 1.45E+08 & 9.94E+07 & 2.86E+04 &  &  &  &  & 2.45E+08 \\
	 & DIME35EB &  &  &  &  & 7.12E+07 & 1.79E+07 & 1.23E+07 & 3.53E+03 &  &  &  &  & 1.01E+08 \\
	 & STYRENE &  &  & 1.68E+07 &  & 1.45E+07 & 1.65E+07 & 1.13E+07 & 3.25E+03 &  &  &  &  & 5.91E+07 \\
	 & BENZAL &  &  &  &  &  & 2.77E+07 & 1.90E+07 & 5.46E+03 &  &  &  &  & 4.68E+07 \\
	 & PHENOL &  &  & 1.86E+07 &  &  &  &  &  &  &  &  &  & 1.86E+07 \\
	\hline Formaldehyde & HCHO & 2.74E+07 & 5.76E+08 &  &  &  & 2.12E+08 & 2.78E+08 & 1.09E+07 &  & 1.23E+09 & 2.22E+07 &  & 2.35E+09 \\ \hline
	\parbox[t]{2mm}{\multirow{9}{*}{\rotatebox[origin=c]{90}{Other Aldehydes}}} & CH3CHO & 2.82E+06 & 7.80E+07 & 7.07E+07 &  &  & 5.74E+07 & 1.15E+08 & 2.09E+06 &  & 2.22E+08 & 5.17E+06 &  & 5.53E+08 \\
	 & C2H5CHO & 1.61E+06 & 5.91E+07 &  &  &  & 9.67E+06 & 1.94E+07 & 3.52E+05 &  & 8.41E+07 & 3.92E+06 &  & 1.78E+08 \\
	 & C3H7CHO & 1.29E+04 & 4.76E+07 &  &  &  &  &  &  &  & 6.78E+07 & 3.16E+06 &  & 1.19E+08 \\
	 & IPRCHO & 1.29E+04 & 4.76E+07 &  &  &  &  &  &  &  & 4.52E+07 & 3.16E+06 &  & 9.60E+07 \\
	 & C4H9CHO & 1.08E+04 & 3.99E+07 &  &  &  &  &  &  &  &  & 2.64E+06 &  & 4.25E+07 \\
	 & ACR & 1.67E+04 & 6.13E+07 &  &  &  & 1.50E+07 & 3.02E+07 & 5.48E+05 &  &  & 4.06E+06 &  & 1.11E+08 \\
	 & MACR & 1.33E+04 & 4.90E+07 &  &  &  &  &  &  &  &  & 3.25E+06 &  & 5.23E+07 \\
	 & C4ALDB & 1.33E+04 & 4.90E+07 &  &  &  & 8.01E+06 & 1.61E+07 & 2.92E+05 &  &  & 3.25E+06 &  & 7.67E+07 \\
	 & MGLYOX &  &  &  &  &  &  &  &  &  & 4.52E+07 &  &  & 4.52E+07 \\
	\hline Alkadienes and & C4H6 & 1.32E+07 & 2.34E+08 & 3.10E+08 & 2.09E+10 &  & 4.51E+08 & 1.21E+08 & 3.14E+07 & 1.98E+07 & 2.84E+08 & 1.98E+07 &  & 2.24E+10 \\
	Other Alkynes & C5H8 & 1.05E+07 & 1.86E+08 &  & 1.66E+10 &  &  &  &  &  &  & 1.58E+07 & 3.11E+09 & 2.00E+10 \\
	\hline \multirow{4}{*}{Organic Acids} & HCOOH & 1.27E+06 & 7.07E+08 &  &  &  &  &  &  &  & 1.67E+08 & 5.23E+07 &  & 9.28E+08 \\
	 & CH3CO2H & 9.72E+05 & 5.42E+08 & 4.37E+07 &  &  &  &  &  &  & 1.28E+08 & 4.01E+07 &  & 7.55E+08 \\
	 & PROPACID & 7.88E+05 & 4.39E+08 &  &  &  &  &  &  &  & 1.04E+08 & 3.25E+07 &  & 5.77E+08 \\
	 & ACO2H &  &  & 3.64E+07 &  &  &  &  &  &  &  &  &  & 3.64E+07 \\
	\hline \parbox[t]{2mm}{\multirow{19}{*}{\rotatebox[origin=c]{90}{Alcohols}}} & CH3OH & 5.18E+04 &  & 2.12E+06 &  & 2.00E+09 &  &  &  &  & 4.03E+07 & 1.81E+07 &  & 2.07E+09 \\*
	 & C2H5OH & 3.60E+04 & 9.73E+08 & 5.98E+07 &  & 2.05E+09 &  &  &  &  & 2.80E+07 & 4.77E+07 &  & 3.16E+09 \\*
	 & NPROPOL & 2.76E+04 &  &  &  & 1.67E+08 &  &  &  &  & 2.15E+07 & 5.78E+06 &  & 1.94E+08 \\*
	 & IPROPOL & 2.76E+04 &  & 7.52E+05 &  & 2.67E+08 &  &  &  &  & 2.15E+07 &  &  & 2.89E+08 \\*
	 & NBUTOL & 2.24E+04 &  &  &  & 1.62E+08 &  &  &  &  & 1.74E+07 &  &  & 1.80E+08 \\
	 & BUT2OL & 2.24E+04 &  &  &  & 1.08E+08 &  &  &  &  & 1.74E+07 & 7.80E+06 &  & 1.34E+08 \\
	 & IBUTOL & 2.24E+04 &  &  &  & 6.77E+07 &  &  &  &  & 1.74E+07 &  &  & 8.51E+07 \\
	 & TBUTOL & 2.24E+04 &  &  &  &  &  &  &  &  & 1.74E+07 &  &  & 1.74E+07 \\
	 & PECOH & 1.88E+04 &  &  &  &  &  &  &  &  & 1.46E+07 &  &  & 1.47E+07 \\
	 & IPEAOH & 1.88E+04 &  &  &  &  &  &  &  &  & 1.46E+07 &  &  & 1.47E+07 \\
	 & ME3BUOL & 1.88E+04 &  &  &  &  &  &  &  &  & 1.46E+07 &  &  & 1.47E+07 \\
	 & IPECOH & 1.88E+04 &  &  &  &  &  &  &  &  & 1.46E+07 &  &  & 1.47E+07 \\
	 & IPEBOH & 1.88E+04 &  &  &  &  &  &  &  &  & 1.46E+07 &  &  & 1.47E+07 \\
	 & CYHEXOL & 1.66E+04 &  &  &  &  &  &  &  &  & 1.29E+07 &  &  & 1.29E+07 \\
	 & MIBKAOH & 1.43E+04 &  &  &  & 3.46E+07 &  &  &  &  & 1.11E+07 &  &  & 4.57E+07 \\
	 & ETHGLY & 2.67E+04 &  &  &  & 4.85E+07 &  &  &  &  & 2.08E+07 &  &  & 6.93E+07 \\
	 & PROPGLY & 2.18E+04 &  &  &  & 9.67E+07 &  &  &  &  & 1.69E+07 &  &  & 1.14E+08 \\
	 & C6H5CH2OH &  &  &  &  & 2.78E+07 &  &  &  &  &  &  &  & 2.78E+07 \\
	 & MBO & 1.93E+04 &  &  &  &  &  &  &  &  & 1.50E+07 &  &  & 1.50E+07 \\
	\hline \parbox[t]{2mm}{\multirow{10}{*}{\rotatebox[origin=c]{90}{Ketones}}} & CH3COCH3 & 1.29E+05 & 1.08E+07 & 1.66E+08 &  & 2.13E+09 & 6.45E+06 & 3.59E+07 &  &  & 1.73E+08 & 1.06E+06 &  & 2.53E+09 \\
	 & MEK &  & 8.73E+06 &  &  & 1.03E+09 &  &  &  &  &  & 8.54E+05 &  & 1.04E+09 \\
	 & MPRK &  & 7.31E+06 &  &  &  &  &  &  &  &  & 7.15E+05 &  & 8.03E+06 \\
	 & DIEK &  & 7.31E+06 &  &  &  &  &  &  &  &  & 7.15E+05 &  & 8.03E+06 \\
	 & MIPK &  & 7.31E+06 &  &  &  &  &  &  &  &  & 7.15E+05 &  & 8.03E+06 \\
	 & HEX2ONE &  & 6.29E+06 &  &  &  &  &  &  &  &  & 6.15E+05 &  & 6.90E+06 \\
	 & HEX3ONE &  & 6.29E+06 &  &  &  &  &  &  &  &  & 6.15E+05 &  & 6.90E+06 \\
	 & MIBK &  & 6.29E+06 &  &  & 6.18E+08 &  &  &  &  &  & 6.15E+05 &  & 6.25E+08 \\
	 & MTBK &  & 6.29E+06 &  &  &  &  &  &  &  &  & 6.15E+05 &  & 6.90E+06 \\
	 & CYHEXONE &  & 6.42E+06 & 8.91E+06 &  & 5.05E+07 &  &  &  &  &  & 6.28E+05 &  & 6.64E+07 \\
	\hline \multirow{3}{*}{Terpenes} & APINENE &  &  &  &  &  &  &  &  &  &  & 2.28E+06 & 3.89E+08 & 3.91E+08 \\
	 & BPINENE &  &  &  &  &  &  &  &  &  &  & 2.28E+06 & 3.89E+08 & 3.91E+08 \\
	 & LIMONENE &  &  &  &  & 6.87E+07 &  &  &  &  &  & 3.42E+06 & 3.89E+08 & 4.61E+08 \\
	\hline \parbox[t]{2mm}{\multirow{6}{*}{\rotatebox[origin=c]{90}{Esters}}} & METHACET &  &  & 6.18E+07 &  &  &  &  &  &  &  &  &  & 6.18E+07 \\
	 & ETHACET &  &  & 7.08E+06 &  & 1.38E+09 &  &  &  &  &  &  &  & 1.39E+09 \\
	 & NBUTACET &  &  &  &  & 9.65E+08 &  &  &  &  &  &  &  & 9.65E+08 \\
	 & IPROACET &  &  &  &  & 3.40E+08 &  &  &  &  &  &  &  & 3.40E+08 \\
	 & CH3OCHO &  &  & 6.93E+06 &  &  &  &  &  &  &  &  &  & 6.93E+06 \\
	 & NPROACET &  &  &  &  & 1.27E+08 &  &  &  &  &  & 5.94E+06 &  & 1.33E+08 \\
	\hline \parbox[t]{2mm}{\multirow{10}{*}{\rotatebox[origin=c]{90}{Ethers}}} & CH3OCH3 &  & 3.36E+07 & 2.43E+08 &  & 7.77E+07 &  &  &  &  &  &  &  & 3.54E+08 \\*
	 & DIETETHER &  & 2.09E+07 & 9.06E+07 &  &  &  &  &  &  &  &  &  & 1.11E+08 \\
	 & MTBE &  & 1.76E+07 &  &  &  &  &  &  &  &  &  &  & 1.76E+07 \\
	 & DIIPRETHER &  & 1.52E+07 & 6.57E+07 &  &  &  &  &  &  &  & 1.47E+07 &  & 9.56E+07 \\
	 & ETBE &  & 1.52E+07 &  &  &  &  &  &  &  &  &  &  & 1.52E+07 \\
	 & MO2EOL &  & 2.04E+07 &  &  & 9.40E+07 &  &  &  &  &  &  &  & 1.14E+08 \\
	 & EOX2EOL &  & 1.72E+07 &  &  & 7.94E+07 &  &  &  &  &  &  &  & 9.66E+07 \\
	 & PR2OHMOX &  & 1.72E+07 &  &  & 1.59E+08 &  &  &  &  &  &  &  & 1.76E+08 \\
	 & BUOX2ETOH &  & 1.31E+07 &  &  & 7.62E+08 &  &  &  &  &  &  &  & 7.75E+08 \\
	 & BOX2PROL &  & 1.17E+07 &  &  &  &  &  &  &  &  &  &  & 1.17E+07 \\
	\hline \parbox[t]{2mm}{\multirow{12}{*}{\rotatebox[origin=c]{90}{Chlorinated Hydrocarbons}}} & CH2CL2 &  &  & 1.75E+08 &  & 6.16E+08 &  &  &  &  &  & 1.09E+06 &  & 7.92E+08 \\
	 & CH3CH2CL &  &  & 1.36E+08 &  &  &  &  &  &  &  &  &  & 1.36E+08 \\
	 & CH3CCL3 &  &  &  &  & 4.31E+08 &  &  &  &  &  & 3.47E+05 &  & 4.31E+08 \\
	 & TRICLETH &  &  & 6.66E+07 &  & 9.75E+08 &  &  &  &  &  & 3.52E+05 &  & 1.04E+09 \\
	 & CDICLETH &  &  & 4.51E+07 &  &  &  &  &  &  &  & 7.11E+05 &  & 4.58E+07 \\
	 & TDICLETH &  &  & 4.51E+07 &  &  &  &  &  &  &  & 4.74E+05 &  & 4.56E+07 \\
	 & CH3CL &  &  & 1.39E+08 &  &  &  &  &  &  &  &  &  & 1.39E+08 \\
	 & CCL2CH2 &  &  & 4.51E+07 &  &  &  &  &  &  &  &  &  & 4.51E+07 \\
	 & CHCL2CH3 &  &  &  &  &  &  &  &  &  &  & 5.35E+05 &  & 5.35E+05 \\
	 & VINCL &  &  & 4.20E+07 &  &  &  &  &  &  &  &  &  & 4.20E+07 \\
	 & TCE &  &  & 1.05E+07 &  & 2.36E+08 &  &  &  &  &  & 6.93E+05 &  & 2.48E+08 \\
	 & CHCL3 &  &  & 2.93E+07 &  &  &  &  &  &  &  &  &  & 2.93E+07 \\
	\hline \multicolumn{2}{c}{Total}  & 4.30E+09 & 1.12E+10 & 3.85E+10 & 4.07E+11 & 2.73E+10 & 6.00E+09 & 2.47E+09 & 2.16E+08 & 1.85E+08 & 6.61E+09 & 8.82E+08 & 4.28E+09 & 5.09E+11 \\
	\hline \hline
\end{longtable}
    \end{landscape}%
}

{
    \begin{landscape}%
        \centering%
        %\tiny
\begin{longtable}{lllllllllllllll}
	\caption{Netherlands AVOC and BVOC emissions, in molecules~cm$^{-2}$~s$^{-1}$, mapped to MCM~v3.2 species.}\\%
	\hline \hline
	\textbf{Type} & \textbf{MCM.species} & \textbf{Snap.1} & \textbf{Snap.2} & \textbf{Snap.34} & \textbf{Snap.5} & \textbf{Snap.6} & \textbf{Snap.71} & \textbf{Snap.72} & \textbf{Snap.73} & \textbf{Snap.74} & \textbf{Snap.8} & \textbf{Snap.9} & \textbf{BVOC} & \textbf{Total}\\
	\endhead
	\hline
	Ethane & C2H6 & 5.70E+08 & 5.15E+08 & 6.20E+09 &  &  & 3.15E+08 & 4.38E+07 & 5.22E+07 &  & 1.36E+08 & 1.28E+08 &  & 7.96E+09 \\
	\hline Propane & C3H8 & 1.60E+09 & 6.01E+08 & 2.47E+09 & 3.38E+10 & 2.93E+08 & 1.53E+07 & 2.99E+07 & 5.22E+08 & 1.66E+07 & 8.83E+07 & 3.96E+07 &  & 3.94E+10 \\ \hline
	\multirow{2}{*}{Butanes} & NC4H10 & 1.13E+09 & 6.33E+08 & 9.48E+08 & 1.42E+11 & 1.09E+09 & 3.41E+08 & 3.09E+07 &  & 2.74E+08 & 8.82E+07 & 2.02E+07 &  & 1.47E+11 \\
	 & IC4H10 & 1.37E+08 & 2.22E+08 & 2.32E+08 & 3.46E+10 & 4.93E+07 & 1.59E+08 & 1.44E+07 &  & 1.28E+08 & 4.41E+07 & 2.02E+07 &  & 3.56E+10 \\
	\hline \multirow{3}{*}{Pentanes} & NC5H12 & 9.18E+08 & 7.80E+08 &  & 1.03E+11 &  & 2.04E+08 & 1.24E+07 &  & 1.38E+08 & 3.69E+07 & 3.79E+06 &  & 1.05E+11 \\
	 & IC5H12 & 3.87E+08 & 4.18E+08 &  & 6.13E+10 &  & 3.96E+08 & 2.41E+07 &  & 2.67E+08 & 7.05E+07 & 3.79E+06 &  & 6.29E+10 \\
	 & NEOP &  &  &  &  &  &  &  &  &  &  & 3.79E+06 &  & 3.79E+06 \\
	\hline \parbox[t]{2mm}{\multirow{14}{*}{\rotatebox[origin=c]{90}{Hexane and Higher Alkanes}}} & NC6H14 & 5.43E+08 & 3.23E+07 & 8.96E+08 & 1.47E+10 & 9.22E+08 & 7.19E+08 & 1.84E+08 &  & 5.11E+07 & 1.63E+08 & 1.32E+06 &  & 1.82E+10 \\
	 & M2PE &  &  & 1.16E+08 & 2.26E+09 & 1.94E+08 &  &  &  &  & 2.71E+08 & 5.65E+05 &  & 2.84E+09 \\
	 & M3PE &  &  & 8.67E+07 & 1.13E+09 & 1.94E+08 &  &  &  &  & 1.63E+08 &  &  & 1.57E+09 \\
	 & NC7H16 & 2.33E+08 & 5.56E+07 & 4.23E+08 & 1.58E+10 & 3.34E+08 & 1.18E+08 & 3.03E+07 &  & 8.41E+06 & 4.66E+07 & 6.64E+06 &  & 1.71E+10 \\
	 & M2HEX &  &  &  &  & 1.25E+08 & 9.20E+07 & 2.36E+07 &  & 6.54E+06 & 6.99E+07 &  &  & 3.17E+08 \\
	 & M3HEX &  &  &  &  & 1.25E+08 & 6.57E+07 & 1.68E+07 &  & 4.67E+06 & 4.66E+07 &  &  & 2.59E+08 \\
	 & M22C4 &  &  &  &  &  &  &  &  &  & 5.42E+07 &  &  & 5.42E+07 \\
	 & M23C4 &  &  &  &  &  &  &  &  &  & 5.42E+07 &  &  & 5.42E+07 \\
	 & NC8H18 &  &  & 1.74E+08 & 1.17E+10 & 3.66E+07 & 1.04E+08 & 2.66E+07 &  & 7.38E+06 & 2.66E+08 & 2.27E+06 &  & 1.23E+10 \\
	 & NC9H20 &  &  & 9.71E+07 &  & 8.80E+08 &  &  &  &  &  & 7.59E+05 &  & 9.78E+08 \\
	 & NC10H22 &  &  & 1.23E+08 &  & 1.70E+09 & 4.63E+07 & 1.19E+07 &  & 3.29E+06 &  & 1.14E+06 &  & 1.89E+09 \\
	 & NC11H24 &  &  & 4.78E+07 &  & 6.95E+08 & 1.69E+07 & 4.32E+06 &  & 1.20E+06 & 2.99E+07 & 4.15E+05 &  & 7.96E+08 \\
	 & NC12H26 &  &  &  &  & 4.91E+07 & 2.75E+08 & 7.05E+07 &  & 1.96E+07 & 2.74E+07 &  &  & 4.42E+08 \\
	 & CHEX &  & 5.15E+07 & 2.96E+07 &  & 1.99E+08 &  &  &  &  &  & 3.86E+05 &  & 2.80E+08 \\
	\hline Ethene & C2H4 & 1.23E+08 & 1.11E+09 & 2.58E+09 &  &  & 1.74E+09 & 5.63E+08 & 2.80E+08 &  & 1.82E+09 & 4.70E+07 &  & 8.25E+09 \\ \hline
	Propene & C3H6 & 8.22E+07 & 3.19E+08 & 1.26E+08 &  &  & 6.10E+08 & 9.38E+07 & 1.24E+08 &  & 3.09E+08 & 1.35E+07 &  & 1.68E+09 \\
	\hline \parbox[t]{2mm}{\multirow{11}{*}{\rotatebox[origin=c]{90}{Higher Alkenes}}} & HEX1ENE & 1.60E+07 & 2.47E+06 &  &  &  &  &  &  &  &  & 5.51E+06 &  & 2.40E+07 \\
	 & BUT1ENE &  & 3.45E+06 & 1.78E+08 &  &  &  &  &  &  & 4.91E+06 &  &  & 1.86E+08 \\
	 & MEPROPENE &  &  &  &  &  &  &  &  &  & 2.46E+06 &  &  & 2.46E+06 \\
	 & TBUT2ENE &  &  &  &  &  &  &  &  &  & 2.46E+06 &  &  & 2.46E+06 \\
	 & CBUT2ENE &  &  &  &  &  &  &  &  &  & 2.46E+06 &  &  & 2.46E+06 \\
	 & CPENT2ENE &  & 1.09E+06 &  &  &  &  &  &  &  & 9.82E+05 &  &  & 2.07E+06 \\
	 & TPENT2ENE &  & 1.09E+06 &  &  &  &  &  &  &  & 9.82E+05 &  &  & 2.07E+06 \\
	 & PENT1ENE &  & 9.87E+05 & 2.56E+05 &  &  &  &  &  &  & 3.93E+06 &  &  & 5.17E+06 \\
	 & ME2BUT2ENE &  & 5.92E+05 &  &  &  &  &  &  &  & 1.96E+06 &  &  & 2.56E+06 \\
	 & ME3BUT1ENE &  & 5.92E+05 &  &  &  &  &  &  &  & 1.96E+06 &  &  & 2.56E+06 \\
	 & ME2BUT1ENE &  & 3.95E+05 &  &  &  &  &  &  &  &  &  &  & 3.95E+05 \\
	\hline Ethyne & C2H2 & 2.00E+06 & 4.29E+08 & 8.16E+07 &  &  & 1.62E+09 & 2.65E+08 & 1.10E+08 & 6.66E+07 & 6.36E+08 & 1.77E+07 &  & 3.22E+09 \\ \hline
	Benzene & BENZENE & 1.18E+08 & 5.17E+08 & 2.56E+08 & 3.58E+09 &  & 3.89E+08 & 3.37E+07 &  & 9.35E+06 & 1.06E+08 & 9.04E+06 &  & 5.02E+09 \\
	\hline Toluene & TOLUENE & 1.29E+08 & 2.10E+08 & 3.57E+07 & 3.04E+09 & 1.97E+09 & 8.80E+08 & 2.14E+07 &  & 7.93E+06 & 6.42E+07 & 6.00E+06 &  & 6.37E+09 \\ \hline
	\multirow{3}{*}{Xylenes} & MXYL & 6.30E+07 & 7.31E+06 & 6.93E+04 & 4.40E+08 & 1.14E+09 & 1.88E+08 & 9.02E+06 &  & 1.25E+06 & 2.62E+07 & 1.22E+06 &  & 1.88E+09 \\
	 & OXYL & 1.40E+07 & 7.31E+06 & 2.77E+04 & 4.40E+08 & 2.85E+08 & 1.88E+08 & 9.02E+06 &  & 1.25E+06 & 2.13E+07 & 7.34E+05 &  & 9.66E+08 \\
	 & PXYL &  & 7.31E+06 & 2.77E+04 & 4.40E+08 & 2.85E+08 & 1.41E+08 & 6.76E+06 &  & 9.38E+05 & 2.62E+07 & 9.79E+05 &  & 9.08E+08 \\
	\hline \multirow{3}{*}{Trimethylbenzenes} & TM123B & 1.06E+04 & 2.03E+05 &  &  & 1.91E+07 & 3.23E+07 &  &  &  & 0.00E+00 & 1.11E+05 &  & 5.17E+07 \\
	 & TM124B & 1.06E+04 & 2.03E+05 & 4.15E+07 &  & 6.51E+07 & 1.35E+08 &  &  &  & 0.00E+00 & 1.49E+05 &  & 2.42E+08 \\
	 & TM135B & 1.06E+04 & 2.03E+05 &  &  & 1.91E+07 & 5.16E+07 &  &  &  & 0.00E+00 & 1.49E+05 &  & 7.11E+07 \\
	\hline \parbox[t]{2mm}{\multirow{11}{*}{\rotatebox[origin=c]{90}{Other Aromatics}}} & EBENZ & 2.15E+07 &  & 4.69E+07 &  & 5.20E+07 & 1.40E+08 & 5.04E+07 & 9.77E+04 &  & 3.16E+08 & 1.32E+06 &  & 6.29E+08 \\
	 & PBENZ &  &  &  &  & 1.15E+07 & 1.24E+08 & 4.45E+07 & 8.63E+04 &  & 5.08E+07 & 5.81E+05 &  & 2.31E+08 \\
	 & IPBENZ &  &  &  &  & 4.21E+07 &  &  &  &  & 5.08E+07 & 5.81E+05 &  & 9.35E+07 \\
	 & PETHTOL &  &  &  &  & 3.83E+06 &  &  &  &  & 1.02E+08 &  &  & 1.05E+08 \\
	 & METHTOL &  &  &  &  & 1.15E+07 &  &  &  &  & 1.02E+08 &  &  & 1.13E+08 \\
	 & OETHTOL &  &  &  &  &  &  &  &  &  & 7.62E+07 &  &  & 7.62E+07 \\
	 & DIET35TOL &  &  &  &  &  & 2.62E+08 & 9.42E+07 & 1.83E+05 &  &  &  &  & 3.56E+08 \\
	 & DIME35EB &  &  &  &  & 6.51E+07 & 3.23E+07 & 1.16E+07 & 2.25E+04 &  &  &  &  & 1.09E+08 \\
	 & STYRENE &  &  & 4.78E+07 &  & 1.32E+07 & 2.98E+07 & 1.07E+07 & 2.08E+04 &  &  &  &  & 1.02E+08 \\
	 & BENZAL &  &  &  &  &  & 5.01E+07 & 1.80E+07 & 3.49E+04 &  &  &  &  & 6.81E+07 \\
	 & PHENOL &  &  & 5.29E+07 &  &  &  &  &  &  &  &  &  & 5.29E+07 \\
	\hline Formaldehyde & HCHO & 1.55E+08 & 1.59E+09 &  &  &  & 3.83E+08 & 2.63E+08 & 6.97E+07 &  & 9.11E+08 & 7.29E+06 &  & 3.38E+09 \\ \hline
	\parbox[t]{2mm}{\multirow{9}{*}{\rotatebox[origin=c]{90}{Other Aldehydes}}} & CH3CHO & 8.96E+06 & 4.73E+07 & 3.05E+06 &  &  & 1.04E+08 & 1.09E+08 & 1.34E+07 &  & 2.48E+08 & 1.69E+06 &  & 5.35E+08 \\
	 & C2H5CHO & 5.10E+06 & 3.59E+07 &  &  &  & 1.75E+07 & 1.84E+07 & 2.25E+06 &  & 9.39E+07 & 1.28E+06 &  & 1.74E+08 \\
	 & C3H7CHO & 2.21E+04 & 2.89E+07 &  &  &  &  &  &  &  & 7.56E+07 & 1.03E+06 &  & 1.06E+08 \\
	 & IPRCHO & 2.21E+04 & 2.89E+07 &  &  &  &  &  &  &  & 5.04E+07 & 1.03E+06 &  & 8.04E+07 \\
	 & C4H9CHO & 1.85E+04 & 2.42E+07 &  &  &  &  &  &  &  &  & 8.64E+05 &  & 2.51E+07 \\
	 & ACR & 2.84E+04 & 3.72E+07 &  &  &  & 2.71E+07 & 2.86E+07 & 3.50E+06 &  &  & 1.33E+06 &  & 9.78E+07 \\
	 & MACR & 2.27E+04 & 2.97E+07 &  &  &  &  &  &  &  &  & 1.06E+06 &  & 3.08E+07 \\
	 & C4ALDB & 2.27E+04 & 2.97E+07 &  &  &  & 1.45E+07 & 1.53E+07 & 1.87E+06 &  &  & 1.06E+06 &  & 6.24E+07 \\
	 & MGLYOX &  &  &  &  &  &  &  &  &  & 5.05E+07 &  &  & 5.05E+07 \\
	\hline Alkadienes and & C4H6 & 2.06E+07 & 1.38E+08 & 4.43E+09 & 2.46E+10 &  & 8.14E+08 & 1.15E+08 & 2.01E+08 & 1.21E+08 & 3.98E+08 & 6.54E+06 &  & 3.08E+10 \\
	Other Alkynes & C5H8 & 1.64E+07 & 1.10E+08 &  & 1.95E+10 &  &  &  &  &  &  & 5.19E+06 &  & 1.97E+10 \\
	\hline \multirow{4}{*}{Organic Acids} & HCOOH & 3.39E+06 & 4.54E+08 &  &  &  &  &  &  &  & 0.00E+00 & 1.71E+07 &  & 4.74E+08 \\
	 & CH3CO2H & 2.60E+06 & 3.48E+08 & 1.24E+08 &  &  &  &  &  &  & 0.00E+00 & 1.31E+07 &  & 4.88E+08 \\
	 & PROPACID & 2.11E+06 & 2.82E+08 &  &  &  &  &  &  &  & 0.00E+00 & 1.06E+07 &  & 2.95E+08 \\
	 & ACO2H &  &  & 1.04E+08 &  &  &  &  &  &  &  &  &  & 1.04E+08 \\
	\hline \parbox[t]{2mm}{\multirow{19}{*}{\rotatebox[origin=c]{90}{Alcohols}}} & CH3OH & 8.82E+04 &  & 1.08E+06 &  & 1.84E+09 &  &  &  &  & 1.19E+05 & 5.92E+06 &  & 1.85E+09 \\*
	 & C2H5OH & 6.14E+04 & 6.51E+08 & 3.05E+07 &  & 1.88E+09 &  &  &  &  & 8.29E+04 & 1.56E+07 &  & 2.58E+09 \\*
	 & NPROPOL & 4.70E+04 &  &  &  & 1.53E+08 &  &  &  &  & 6.35E+04 & 1.89E+06 &  & 1.55E+08 \\*
	 & IPROPOL & 4.70E+04 &  & 3.83E+05 &  & 2.45E+08 &  &  &  &  & 6.35E+04 &  &  & 2.46E+08 \\*
	 & NBUTOL & 3.81E+04 &  &  &  & 1.49E+08 &  &  &  &  & 5.15E+04 &  &  & 1.49E+08 \\
	 & BUT2OL & 3.81E+04 &  &  &  & 9.94E+07 &  &  &  &  & 5.15E+04 & 2.56E+06 &  & 1.02E+08 \\
	 & IBUTOL & 3.81E+04 &  &  &  & 6.21E+07 &  &  &  &  & 5.15E+04 &  &  & 6.22E+07 \\
	 & TBUTOL & 3.81E+04 &  &  &  &  &  &  &  &  & 5.15E+04 &  &  & 8.97E+04 \\
	 & PECOH & 3.21E+04 &  &  &  &  &  &  &  &  & 4.33E+04 &  &  & 7.54E+04 \\
	 & IPEAOH & 3.21E+04 &  &  &  &  &  &  &  &  & 4.33E+04 &  &  & 7.54E+04 \\
	 & ME3BUOL & 3.21E+04 &  &  &  &  &  &  &  &  & 4.33E+04 &  &  & 7.54E+04 \\
	 & IPECOH & 3.21E+04 &  &  &  &  &  &  &  &  & 4.33E+04 &  &  & 7.54E+04 \\
	 & IPEBOH & 3.21E+04 &  &  &  &  &  &  &  &  & 4.33E+04 &  &  & 7.54E+04 \\
	 & CYHEXOL & 2.82E+04 &  &  &  &  &  &  &  &  & 3.81E+04 &  &  & 6.64E+04 \\
	 & MIBKAOH & 2.43E+04 &  &  &  & 3.17E+07 &  &  &  &  & 3.29E+04 &  &  & 3.18E+07 \\
	 & ETHGLY & 4.56E+04 &  &  &  & 4.45E+07 &  &  &  &  & 6.15E+04 &  &  & 4.46E+07 \\
	 & PROPGLY & 3.72E+04 &  &  &  & 8.88E+07 &  &  &  &  & 5.02E+04 &  &  & 8.88E+07 \\
	 & C6H5CH2OH &  &  &  &  & 2.55E+07 &  &  &  &  &  &  &  & 2.55E+07 \\
	 & MBO & 3.28E+04 &  &  &  &  &  &  &  &  & 4.43E+04 &  &  & 7.72E+04 \\
	\hline \parbox[t]{2mm}{\multirow{10}{*}{\rotatebox[origin=c]{90}{Ketones}}} & CH3COCH3 & 2.19E+05 & 4.54E+06 & 4.72E+08 &  & 1.91E+09 & 1.16E+07 & 3.40E+07 &  &  & 1.11E+08 & 3.54E+05 &  & 2.54E+09 \\
	 & MEK &  & 3.65E+06 &  &  & 9.22E+08 &  &  &  &  &  & 2.85E+05 &  & 9.26E+08 \\
	 & MPRK &  & 3.06E+06 &  &  &  &  &  &  &  &  & 2.39E+05 &  & 3.30E+06 \\
	 & DIEK &  & 3.06E+06 &  &  &  &  &  &  &  &  & 2.39E+05 &  & 3.30E+06 \\
	 & MIPK &  & 3.06E+06 &  &  &  &  &  &  &  &  & 2.39E+05 &  & 3.30E+06 \\
	 & HEX2ONE &  & 2.63E+06 &  &  &  &  &  &  &  &  & 2.05E+05 &  & 2.84E+06 \\
	 & HEX3ONE &  & 2.63E+06 &  &  &  &  &  &  &  &  & 2.05E+05 &  & 2.84E+06 \\
	 & MIBK &  & 2.63E+06 &  &  & 5.53E+08 &  &  &  &  &  & 2.05E+05 &  & 5.56E+08 \\
	 & MTBK &  & 2.63E+06 &  &  &  &  &  &  &  &  & 2.05E+05 &  & 2.84E+06 \\
	 & CYHEXONE &  & 2.69E+06 & 2.54E+07 &  & 4.51E+07 &  &  &  &  &  & 2.09E+05 &  & 7.34E+07 \\
	\hline \multirow{3}{*}{Terpenes} & APINENE &  &  &  &  &  &  &  &  &  &  & 7.70E+05 & 1.35E+08 & 1.36E+08 \\
	 & BPINENE &  &  &  &  &  &  &  &  &  &  & 7.70E+05 & 1.35E+08 & 1.36E+08 \\
	 & LIMONENE &  &  &  &  & 6.31E+07 &  &  &  &  &  & 1.16E+06 & 1.35E+08 & 2.00E+08 \\
	\hline \parbox[t]{2mm}{\multirow{6}{*}{\rotatebox[origin=c]{90}{Esters}}} & METHACET &  &  & 2.67E+06 &  &  &  &  &  &  &  &  &  & 2.67E+06 \\
	 & ETHACET &  &  & 3.06E+05 &  & 1.29E+09 &  &  &  &  &  &  &  & 1.29E+09 \\
	 & NBUTACET &  &  &  &  & 9.03E+08 &  &  &  &  &  &  &  & 9.03E+08 \\
	 & IPROACET &  &  &  &  & 3.18E+08 &  &  &  &  &  &  &  & 3.18E+08 \\
	 & CH3OCHO &  &  & 2.99E+05 &  &  &  &  &  &  &  &  &  & 2.99E+05 \\
	 & NPROACET &  &  &  &  & 1.19E+08 &  &  &  &  &  & 2.01E+06 &  & 1.21E+08 \\
	\hline \parbox[t]{2mm}{\multirow{10}{*}{\rotatebox[origin=c]{90}{Ethers}}} & CH3OCH3 &  & 2.58E+07 & 1.05E+07 &  & 7.08E+07 &  &  &  &  &  &  &  & 1.07E+08 \\*
	 & DIETETHER &  & 1.60E+07 & 3.91E+06 &  &  &  &  &  &  &  &  &  & 1.99E+07 \\
	 & MTBE &  & 1.35E+07 &  &  &  &  &  &  &  &  &  &  & 1.35E+07 \\
	 & DIIPRETHER &  & 1.16E+07 & 2.84E+06 &  &  &  &  &  &  &  & 4.82E+06 &  & 1.93E+07 \\
	 & ETBE &  & 1.16E+07 &  &  &  &  &  &  &  &  &  &  & 1.16E+07 \\
	 & MO2EOL &  & 1.56E+07 &  &  & 8.57E+07 &  &  &  &  &  &  &  & 1.01E+08 \\
	 & EOX2EOL &  & 1.32E+07 &  &  & 7.24E+07 &  &  &  &  &  &  &  & 8.56E+07 \\
	 & PR2OHMOX &  & 1.32E+07 &  &  & 1.45E+08 &  &  &  &  &  &  &  & 1.58E+08 \\
	 & BUOX2ETOH &  & 1.01E+07 &  &  & 6.95E+08 &  &  &  &  &  &  &  & 7.05E+08 \\
	 & BOX2PROL &  & 8.99E+06 &  &  &  &  &  &  &  &  &  &  & 8.99E+06 \\
	\hline \parbox[t]{2mm}{\multirow{12}{*}{\rotatebox[origin=c]{90}{Chlorinated Hydrocarbons}}} & CH2CL2 &  &  & 4.99E+08 &  & 5.24E+08 &  &  &  &  &  & 3.68E+05 &  & 1.02E+09 \\
	 & CH3CH2CL &  &  & 3.86E+08 &  &  &  &  &  &  &  &  &  & 3.86E+08 \\
	 & CH3CCL3 &  &  &  &  & 3.67E+08 &  &  &  &  &  & 1.17E+05 &  & 3.67E+08 \\
	 & TRICLETH &  &  & 1.90E+08 &  & 8.30E+08 &  &  &  &  &  & 1.19E+05 &  & 1.02E+09 \\
	 & CDICLETH &  &  & 1.28E+08 &  &  &  &  &  &  &  & 2.40E+05 &  & 1.29E+08 \\
	 & TDICLETH &  &  & 1.28E+08 &  &  &  &  &  &  &  & 1.60E+05 &  & 1.29E+08 \\
	 & CH3CL &  &  & 3.95E+08 &  &  &  &  &  &  &  &  &  & 3.95E+08 \\
	 & CCL2CH2 &  &  & 1.28E+08 &  &  &  &  &  &  &  &  &  & 1.28E+08 \\
	 & CHCL2CH3 &  &  &  &  &  &  &  &  &  &  & 1.80E+05 &  & 1.80E+05 \\
	 & VINCL &  &  & 1.20E+08 &  &  &  &  &  &  &  &  &  & 1.20E+08 \\
	 & TCE &  &  & 3.00E+07 &  & 2.01E+08 &  &  &  &  &  & 2.34E+05 &  & 2.32E+08 \\
	 & CHCL3 &  &  & 8.35E+07 &  &  &  &  &  &  &  &  &  & 8.35E+07 \\
	\hline \multicolumn{2}{c}{Total}  & 6.30E+09 & 9.93E+09 & 2.26E+10 & 4.72E+11 & 2.46E+10 & 1.08E+10 & 2.34E+09 & 1.38E+09 & 1.13E+09 & 7.33E+09 & 4.46E+08 & 4.06E+08 & 5.59E+11 \\
	\hline \hline
	\label{t:Netherlands_MCM_emissions}
\end{longtable}
    \end{landscape}%
}

{
    \begin{landscape}%
        \centering%
        %\input{Luxembourg_MCM_emissions}
    \end{landscape}%
}

{%supplement
    \centering%
    %\footnotesize
\begin{longtable}{llllll}
	\caption{Benelux AVOC and BVOC emissions, in molecules~cm$^{-2}$~s$^{-1}$, mapped from MCM~v3.2 species into corresponding MOZART-4 species. Emissions were weighted by the carbon numbers of the respective species.}\\%
	\hline \hline
	\textbf{Type} & \textbf{MCM Species} & \textbf{Belgium} & \textbf{Netherlands} & \textbf{Luxembourg} & \textbf{Total}\\
	\endhead
	\hline
	Ethane & C2H6 & 4.91E+09 & 8.58E+08 & 7.96E+09 & 1.37E+10 \\
	\hline Propane & C3H8 & 3.35E+10 & 4.00E+10 & 3.94E+10 & 1.13E+11 \\ \hline
	\multirow{2}{*}{Butanes} & NC4H10 & 1.00E+11 & 2.79E+11 & 1.17E+11 & 4.96E+11 \\
	 & IC4H10 & 2.42E+10 & 6.80E+10 & 2.85E+10 & 1.21E+11 \\
	\hline \multirow{3}{*}{Pentanes} & NC5H12 & 8.89E+10 & 2.65E+11 & 1.05E+11 & 4.59E+11 \\
	 & IC5H12 & 5.33E+10 & 1.60E+11 & 6.29E+10 & 2.76E+11 \\
	 & NEOP & 1.11E+07 & 0.00E+00 & 3.79E+06 & 1.49E+07 \\
	\hline \parbox[t]{2mm}{\multirow{14}{*}{\rotatebox[origin=c]{90}{Hexane and Higher Alkanes}}} & NC6H14 & 1.81E+10 & 4.92E+10 & 2.18E+10 & 8.91E+10 \\
	 & M2PE & 2.85E+09 & 7.54E+09 & 3.41E+09 & 1.38E+10 \\
	 & M3PE & 1.59E+09 & 3.94E+09 & 1.89E+09 & 7.42E+09 \\
	 & NC7H16 & 2.02E+10 & 5.77E+10 & 2.39E+10 & 1.02E+11 \\
	 & M2HEX & 3.69E+08 & 6.84E+08 & 4.44E+08 & 1.50E+09 \\
	 & M3HEX & 3.18E+08 & 5.45E+08 & 3.63E+08 & 1.23E+09 \\
	 & M22C4 & 4.16E+07 & 6.34E+07 & 6.51E+07 & 1.70E+08 \\
	 & M23C4 & 4.16E+07 & 6.34E+07 & 6.51E+07 & 1.70E+08 \\
	 & NC8H18 & 1.67E+10 & 4.89E+10 & 1.97E+10 & 8.53E+10 \\
	 & NC9H20 & 1.87E+09 & 1.93E+09 & 1.76E+09 & 5.56E+09 \\
	 & NC10H22 & 4.04E+09 & 4.42E+09 & 3.78E+09 & 1.22E+10 \\
	 & NC11H24 & 1.85E+09 & 2.04E+09 & 1.75E+09 & 5.64E+09 \\
	 & NC12H26 & 7.28E+08 & 2.13E+09 & 1.06E+09 & 3.92E+09 \\
	 & CHEX & 3.30E+08 & 2.93E+08 & 3.36E+08 & 9.59E+08 \\
	\hline Ethene & C2H4 & 3.66E+10 & 7.03E+09 & 8.25E+09 & 5.19E+10 \\ \hline
	Propene & C3H6 & 1.82E+09 & 1.68E+09 & 1.68E+09 & 5.18E+09 \\
	\hline \parbox[t]{2mm}{\multirow{11}{*}{\rotatebox[origin=c]{90}{Higher Alkenes}}} & HEX1ENE & 5.13E+07 & 7.55E+05 & 3.60E+07 & 8.81E+07 \\
	 & BUT1ENE & 9.99E+07 & 7.04E+05 & 1.86E+08 & 2.87E+08 \\
	 & MEPROPENE & 9.80E+06 & 0.00E+00 & 2.46E+06 & 1.23E+07 \\
	 & TBUT2ENE & 9.80E+06 & 0.00E+00 & 2.46E+06 & 1.23E+07 \\
	 & CBUT2ENE & 9.80E+06 & 0.00E+00 & 2.46E+06 & 1.23E+07 \\
	 & CPENT2ENE & 1.20E+07 & 2.77E+05 & 2.58E+06 & 1.49E+07 \\
	 & TPENT2ENE & 1.20E+07 & 2.77E+05 & 2.58E+06 & 1.49E+07 \\
	 & PENT1ENE & 3.34E+07 & 2.52E+05 & 6.47E+06 & 4.01E+07 \\
	 & ME2BUT2ENE & 1.37E+07 & 1.51E+05 & 3.20E+06 & 1.71E+07 \\
	 & ME3BUT1ENE & 1.37E+07 & 1.51E+05 & 3.20E+06 & 1.71E+07 \\
	 & ME2BUT1ENE & 2.57E+06 & 1.01E+05 & 4.93E+05 & 3.16E+06 \\
	\hline Ethyne & C2H2 & 2.78E+09 & 4.51E+09 & 3.22E+09 & 1.05E+10 \\ \hline
	Benzene & BENZENE & 3.87E+09 & 9.05E+09 & 4.30E+09 & 1.72E+10 \\
	\hline Toluene & TOLUENE & 5.63E+09 & 1.22E+10 & 6.37E+09 & 2.42E+10 \\ \hline
	\multirow{3}{*}{Xylenes} & MXYL & 2.07E+09 & 3.43E+09 & 2.14E+09 & 7.64E+09 \\
	 & OXYL & 9.60E+08 & 2.16E+09 & 1.10E+09 & 4.22E+09 \\
	 & PXYL & 9.22E+08 & 2.08E+09 & 1.04E+09 & 4.04E+09 \\
	\hline \multirow{3}{*}{Trimethylbenzenes} & TM123B & 5.59E+07 & 9.47E+07 & 6.65E+07 & 2.17E+08 \\
	 & TM124B & 2.19E+08 & 3.72E+08 & 3.12E+08 & 9.03E+08 \\
	 & TM135B & 6.99E+07 & 1.32E+08 & 9.14E+07 & 2.93E+08 \\
	\hline \parbox[t]{2mm}{\multirow{11}{*}{\rotatebox[origin=c]{90}{Other Aromatics}}} & EBENZ & 4.52E+08 & 9.46E+08 & 7.19E+08 & 2.12E+09 \\
	 & PBENZ & 2.03E+08 & 5.95E+08 & 2.97E+08 & 1.10E+09 \\
	 & IPBENZ & 9.73E+07 & 1.34E+08 & 1.20E+08 & 3.51E+08 \\
	 & PETHTOL & 7.72E+07 & 1.34E+08 & 1.36E+08 & 3.47E+08 \\
	 & METHTOL & 8.79E+07 & 1.47E+08 & 1.45E+08 & 3.80E+08 \\
	 & OETHTOL & 5.39E+07 & 9.61E+07 & 9.80E+07 & 2.48E+08 \\
	 & DIET35TOL & 3.84E+08 & 1.32E+09 & 5.60E+08 & 2.26E+09 \\
	 & DIME35EB & 1.45E+08 & 2.68E+08 & 1.56E+08 & 5.69E+08 \\
	 & STYRENE & 7.59E+07 & 1.45E+08 & 1.31E+08 & 3.52E+08 \\
	 & BENZAL & 6.01E+07 & 2.07E+08 & 8.76E+07 & 3.55E+08 \\
	 & PHENOL & 1.59E+07 & 0.00E+00 & 4.54E+07 & 6.13E+07 \\
	\hline Formaldehyde & HCHO & 2.35E+09 & 3.04E+09 & 3.38E+09 & 8.77E+09 \\ \hline
	\parbox[t]{2mm}{\multirow{9}{*}{\rotatebox[origin=c]{90}{Other Aldehydes}}} & CH3CHO & 5.53E+08 & 8.88E+08 & 5.35E+08 & 1.98E+09 \\
	 & C2H5CHO & 2.67E+08 & 2.95E+08 & 2.61E+08 & 8.23E+08 \\
	 & C3H7CHO & 2.37E+08 & 1.34E+08 & 2.11E+08 & 5.82E+08 \\
	 & IPRCHO & 1.92E+08 & 9.14E+07 & 1.61E+08 & 4.44E+08 \\
	 & C4H9CHO & 1.06E+08 & 6.13E+06 & 6.27E+07 & 1.75E+08 \\
	 & ACR & 8.33E+07 & 1.35E+08 & 7.33E+07 & 2.92E+08 \\
	 & MACR & 5.23E+07 & 3.01E+06 & 3.08E+07 & 8.61E+07 \\
	 & C4ALDB & 7.67E+07 & 9.70E+07 & 6.24E+07 & 2.36E+08 \\
	 & MGLYOX & 4.52E+07 & 4.28E+07 & 5.05E+07 & 1.39E+08 \\
	\hline Alkadienes and & C4H6 & 1.79E+10 & 5.46E+10 & 2.47E+10 & 9.72E+10 \\
	Other Alkynes & C5H8 & 2.00E+10 & 6.25E+10 & 1.97E+10 & 1.02E+11 \\
	\hline \multirow{4}{*}{Organic Acids} & HCOOH & 9.28E+08 & 4.04E+07 & 4.74E+08 & 1.44E+09 \\
	 & CH3CO2H & 7.55E+08 & 3.10E+07 & 4.88E+08 & 1.27E+09 \\
	 & PROPACID & 8.65E+08 & 3.77E+07 & 4.42E+08 & 1.34E+09 \\
	 & ACO2H & 5.46E+07 & 0.00E+00 & 1.56E+08 & 2.11E+08 \\
	\hline \parbox[t]{2mm}{\multirow{19}{*}{\rotatebox[origin=c]{90}{Alcohols}}} & CH3OH & 2.07E+09 & 2.40E+09 & 1.85E+09 & 6.32E+09 \\*
	 & C2H5OH & 3.16E+09 & 2.51E+09 & 2.58E+09 & 8.25E+09 \\*
	 & NPROPOL & 2.91E+08 & 3.00E+08 & 2.33E+08 & 8.24E+08 \\*
	 & IPROPOL & 4.34E+08 & 4.79E+08 & 3.69E+08 & 1.28E+09 \\*
	 & NBUTOL & 3.60E+08 & 3.89E+08 & 2.98E+08 & 1.05E+09 \\
	 & BUT2OL & 2.67E+08 & 2.59E+08 & 2.04E+08 & 7.30E+08 \\
	 & IBUTOL & 1.70E+08 & 1.62E+08 & 1.24E+08 & 4.56E+08 \\
	 & TBUTOL & 3.48E+07 & 0.00E+00 & 1.79E+05 & 3.50E+07 \\
	 & PECOH & 3.66E+07 & 0.00E+00 & 1.88E+05 & 3.68E+07 \\
	 & IPEAOH & 3.66E+07 & 0.00E+00 & 1.88E+05 & 3.68E+07 \\
	 & ME3BUOL & 3.66E+07 & 0.00E+00 & 1.88E+05 & 3.68E+07 \\
	 & IPECOH & 3.66E+07 & 0.00E+00 & 1.88E+05 & 3.68E+07 \\
	 & IPEBOH & 3.66E+07 & 0.00E+00 & 1.88E+05 & 3.68E+07 \\
	 & CYHEXOL & 3.87E+07 & 0.00E+00 & 1.99E+05 & 3.89E+07 \\
	 & MIBKAOH & 1.37E+08 & 1.24E+08 & 9.53E+07 & 3.56E+08 \\
	 & ETHGLY & 6.93E+07 & 5.80E+07 & 4.46E+07 & 1.72E+08 \\
	 & PROPGLY & 1.71E+08 & 1.73E+08 & 1.33E+08 & 4.77E+08 \\
	 & C6H5CH2OH & 9.75E+07 & 1.17E+08 & 8.94E+07 & 3.04E+08 \\
	 & MBO & 3.75E+07 & 0.00E+00 & 1.93E+05 & 3.77E+07 \\
	\hline \parbox[t]{2mm}{\multirow{10}{*}{\rotatebox[origin=c]{90}{Ketones}}} & CH3COCH3 & 2.53E+09 & 2.75E+09 & 2.54E+09 & 7.82E+09 \\
	 & MEK & 1.04E+09 & 1.20E+09 & 9.26E+08 & 3.17E+09 \\
	 & MPRK & 1.00E+07 & 4.69E+05 & 4.12E+06 & 1.46E+07 \\
	 & DIEK & 1.00E+07 & 4.69E+05 & 4.12E+06 & 1.46E+07 \\
	 & MIPK & 1.00E+07 & 4.69E+05 & 4.12E+06 & 1.46E+07 \\
	 & HEX2ONE & 1.04E+07 & 4.84E+05 & 4.25E+06 & 1.51E+07 \\
	 & HEX3ONE & 1.04E+07 & 4.84E+05 & 4.25E+06 & 1.51E+07 \\
	 & MIBK & 9.38E+08 & 1.08E+09 & 8.34E+08 & 2.85E+09 \\
	 & MTBK & 1.04E+07 & 4.84E+05 & 4.25E+06 & 1.51E+07 \\
	 & CYHEXONE & 9.97E+07 & 8.83E+07 & 1.10E+08 & 2.98E+08 \\
	\hline \multirow{3}{*}{Terpenes} & APINENE & 3.91E+08 & 1.19E+09 & 1.36E+08 & 1.72E+09 \\
	 & BPINENE & 3.91E+08 & 1.19E+09 & 1.36E+08 & 1.72E+09 \\
	 & LIMONENE & 4.61E+08 & 1.26E+09 & 2.00E+08 & 1.92E+09 \\
	\hline \parbox[t]{2mm}{\multirow{6}{*}{\rotatebox[origin=c]{90}{Esters}}} & METHACET & 3.71E+07 & 0.00E+00 & 1.60E+06 & 3.87E+07 \\*
	 & ETHACET & 1.11E+09 & 1.35E+09 & 1.03E+09 & 3.49E+09 \\*
	 & NBUTACET & 1.16E+09 & 1.41E+09 & 1.08E+09 & 3.65E+09 \\*
	 & IPROACET & 3.40E+08 & 4.14E+08 & 3.18E+08 & 1.07E+09 \\*
	 & CH3OCHO & 6.93E+06 & 0.00E+00 & 2.99E+05 & 7.23E+06 \\*
	 & NPROACET & 1.33E+08 & 1.55E+08 & 1.21E+08 & 4.09E+08 \\*
	\hline \parbox[t]{2mm}{\multirow{10}{*}{\rotatebox[origin=c]{90}{Ethers}}} & CH3OCH3 & 1.42E+08 & 3.72E+07 & 4.28E+07 & 2.22E+08 \\
	 & DIETETHER & 8.92E+07 & 1.17E+06 & 1.60E+07 & 1.06E+08 \\
	 & MTBE & 1.76E+07 & 1.23E+06 & 1.35E+07 & 3.23E+07 \\
	 & DIIPRETHER & 1.15E+08 & 1.27E+06 & 2.32E+07 & 1.39E+08 \\
	 & ETBE & 1.82E+07 & 1.27E+06 & 1.40E+07 & 3.35E+07 \\
	 & MO2EOL & 6.86E+07 & 6.67E+07 & 6.08E+07 & 1.96E+08 \\
	 & EOX2EOL & 7.73E+07 & 7.51E+07 & 6.85E+07 & 2.21E+08 \\
	 & PR2OHMOX & 1.41E+08 & 1.49E+08 & 1.26E+08 & 4.16E+08 \\
	 & BUOX2ETOH & 9.30E+08 & 1.07E+09 & 8.46E+08 & 2.85E+09 \\
	 & BOX2PROL & 1.64E+07 & 1.15E+06 & 1.26E+07 & 3.02E+07 \\
	\hline \parbox[t]{2mm}{\multirow{12}{*}{\rotatebox[origin=c]{90}{Chlorinated Hydrocarbons}}} & CH2CL2 & 1.58E+08 & 8.16E+07 & 2.05E+08 & 4.45E+08 \\
	 & CH3CH2CL & 5.42E+07 & 0.00E+00 & 1.54E+08 & 2.08E+08 \\
	 & CH3CCL3 & 1.73E+08 & 1.14E+08 & 1.47E+08 & 4.34E+08 \\
	 & TRICLETH & 4.17E+08 & 2.58E+08 & 4.08E+08 & 1.08E+09 \\
	 & CDICLETH & 1.83E+07 & 0.00E+00 & 5.15E+07 & 6.98E+07 \\
	 & TDICLETH & 1.82E+07 & 0.00E+00 & 5.15E+07 & 6.97E+07 \\
	 & CH3CL & 2.77E+07 & 0.00E+00 & 7.90E+07 & 1.07E+08 \\
	 & CCL2CH2 & 1.80E+07 & 0.00E+00 & 5.14E+07 & 6.94E+07 \\
	 & CHCL2CH3 & 2.14E+05 & 0.00E+00 & 7.22E+04 & 2.86E+05 \\
	 & VINCL & 1.68E+07 & 0.00E+00 & 4.78E+07 & 6.46E+07 \\
	 & TCE & 9.91E+07 & 6.27E+07 & 9.26E+07 & 2.54E+08 \\
	 & CHCL3 & 5.86E+06 & 0.00E+00 & 1.67E+07 & 2.26E+07 \\
	\hline \multicolumn{2}{c}{Total}  & 4.94E+11 & 1.18E+12 & 5.39E+11 & 2.21E+12 \\
	\hline \hline
	\label{t:MOZART_NMVOC_emissions}
\end{longtable}
}
%The total MCM~v3.2 emissions for each initial species in Tables \ref{t:Belgium_MCM_emissions}, \ref{t:Netherlands_MCM_emissions} and \ref{t:Luxembourg_MCM_emissions} were translated to emissions of MOZART-4 species by weighting with the carbon numbers. The final emissions of the MOZART-4 species representing each MCM~v3.2 species are presented in Table \ref{t:MOZART_NMVOC_emissions} for each country in the Benelux region.
%%%%End Supplement

\subsection{Vertical Mixing with Diurnal Boundary Layer Height} \label{ss:vertical_mixing}
%most likely include in supplement
\begin{itemize}
    \item The MECCA box model used in \citet{Coates:2015} included a constant boundary layer height of $1$~km and no interactions (vertical mixing) with the free troposphere.
    \item The planetary boundary layer (PBL) height varies diurnally and affects chemistry by diluting emissions after sunrise when the PBL rises. The expansion of the PBL into the free troposphere introduces vertical mixing with those chemical species present in the free troposphere. When the PBL collapses in the evening, pollutants are trapped in the PBL.
    \item The mixing layer height was measured as part of the BAERLIN campaign \todo{Reference Boris' paper} over the city of Berlin, Germany. The profile of mean mixing layer height during the campaign period (June -- August 2014) was used in the model to represent the diurnal cycle of the mixing layer height.
    \item The concentrations of the chemical species within the PBL are diluted due to the larger mixing volume when the PBL height increases at the beginning of the day, also the increasing PBL height mixes the chemical species from the free troposphere with the chemical species within the PBL i.e. vertical mixing. The PBL height collapses during night leaving the stable nocturnal boundary layer, trapping the chemical species into a smaller volume thus increasing the concentrations of the chemical species.
    \item This vertical mixing scheme was implemented into the boxmodel using the same approach of \citet{Lourens:2012}.
    \item The mixing ratios of O3, CO and CH4 in the free troposphere were respectively set to $50$ ppbv, $116$ ppbv and $1.8$ ppmv. These conditions were taken from the MATCH-MPIC chemical weather forecast model on the 21st March (the start date of the simulations). The model results (\url{http://cwf.iass-potsdam.de/}) at the 700 hPa height were chosen and the daily average was used as input into the boxmodel. \todo{check reference}
\end{itemize}
