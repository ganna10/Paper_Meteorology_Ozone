\subsection{Ozone Contours} \label{d:ozone_contours}

The ozone contours of Fig.\ref{f:ozone_contours} illustrate the non-linear relationship of ozone mixing ratios to \ce{NO_x} and temperature which is similar to the contours of ozone production to \ce{NO_x} mixing ratios and temperature demonstrated in \citet{Pusede:2014}.
The relationship determined in \citet{Pusede:2014} is relevant to the San Joaquin Valley, California and was inferred using observational methods.

As highlighted by the review of \citet{Pusede:2015} the dependence of ozone to temperature is multi-faceted and is influenced by increased emissions, in particular BVOC such as isoprene emitted from biogenic species, and increased rates of chemical reactions.
The increase in isoprene emissions with temperature is shown to be the more important than the increase of ozone production chemistry through increased reaction rates in Fig.~\ref{f:ozone_contours}, as here there is a large increase in ozone at higher temperatures with increased isoprene emissions alone rather than the increased chemistry.

In order to determine how important the effect of increasing isoprene emissions with temperature are in relation to increases in ozone production chemistry, we repeated the sets of model runs at each \ce{NO_x} condition and temperature using a constant source of isoprene equal to that emitted at the lowest temperature ($288$~K) and then the highest temperature ($313$~K). 
The results of these model runs show that \ldots

The increase in ozone with temperature tends to follow the same shape as the increase of isoprene emissions with temperature (Fig.~\ref{f:isoprene_emissions}), showing that the increase of isoprene emissions with temperature may be the dominant factor of increased ozone with temperature.

\subsection{Ozone Budgets} \label{d:ozone_budgets}

The reaction of \ce{CH3CO3} with NO has the highest contribution to the total \ce{O_x} budget at higher temperatures in each \ce{NO_x} condition in Fig.~\ref{f:ozone_budgets}.
The main source of \ce{CH3CO3} is its chemical equilibrium with PAN, in the presence of \ce{NO2}, other than this equilibrium it is the degradation of acetaldehyde (\ce{CH3CHO}) that leads to a net source of \ce{CH3CO3}.

The increased contribution of \ce{CH3CO3} reaction with NO to \ce{O_x} production budget is more pronounced in the CB05 and RADM2 chemical mechanisms.
Acetaldehyde is included in the NMVOC emissions of each chemical mechanism, however it also has a chemical source from the secondary degradation on many other NMVOC.
An increased chemical source of \ce{CH3CHO} from the representation of tropospheric degradation chemistry in CB05 and RADM2 could be the cause of the higher ozone mixing ratios when using CB05 and RADM2 chemical mechanisms.

\citet{Coates:2015} have shown that the secondary degradation of the RADM2 species HC3 under-estimates the yields of less-reactive ketones at the expense of increased aldehyde yields.
Thus the increased aldehyde yields propagate ozone production through the reactions of the degradation products \ce{CH3CO3} and the methyl peroxy radical (\ce{CH3O2}) with NO.
Moreover, in the CB05, the are no ketone species present in this chemical mechanism and the chemical source of carbonyl species is mainly in the form of aldehydes, leading to a similar propagation of ozone production as described for RADM2.
Thus the underestimation or missing representation of the yield of ketone species in the RADM2 and CB05 chemical mechanisms leads to higher ozone production and the larger ozone mixing ratios seen in Fig.~\ref{f:ozone_contours}.

The increased amounts of \ce{CH3CO3} simulated with RADM2 and CB05 also influences another well-known aspect of the relationship of ozone with temperature, namely that at higher temperatures peroxy nitrates (RO2NO2), such as PAN, are no longer a suitable reservoir of peroxy radicals and \ce{NO_x} due to an increase in PAN decomposition rates with temperature, leading to re-release of peroxy radicals and \ce{NO2} which can then go on to further produce ozone.
As the chemistry of RADM2 and CB05 produces more \ce{CH3CO3} this leads to an increase in the  mixing ratios of RO2NO2, mainly PAN in RADM2 and CB05, and so the equilibrium state is shifted in these chemical mechanisms compared to the MCM~v3.2, CRI~v2 and MOZART-4.
This shift in equilibrium state is another pathway for increased ozone production with temperature in the CB05 and RADM2.

\subsection{Rate of Change of Ozone with Temperature} \label{d:m_O3-T}
