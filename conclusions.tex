In this study, we determined the effects of temperature on ozone production using a box model over a range of temperatures and \ce{NO_x} conditions with a temperature-independent and temperature-dependent source of isoprene emissions.
These simulations were repeated using reduced chemical mechanism schemes (CRIv2, MOZART-4, CB05 and RADM2) typically used in 3D models and compared to the near-explicit MCMv3.2 chemical mechanism.

Each chemical mechanism produced a non-linear relationship of ozone with temperature and \ce{NO_x} with the most ozone produced at high temperatures and moderate emissions of \ce{NO_x}.
Conversely, lower \ce{NO_x} levels led to a minimal increase of ozone at all temperatures.
Thus air quality in a future with higher temperatures predicted with climate change would benefit from dramatical reductions in \ce{NO_x} emissions.

Faster reaction rates at higher temperatures were responsible for a greater absolute increase in ozone than increased isoprene emissions.
The increase in VOC loss with temperature was the dominant process increasing ozone production with temperature.
The increase of ozone with temperature is coupled with increasing OH with temperature increasing VOC loss at higher temperatures.
Enhanced VOC loss at higher temperatures increased the production of peroxy radicals, leading to ozone production.

The rate of change of ozone with temperature using observational data (ERA-Interim) over Europe was twice as high as when using the box model.
This was consistent with the box model not representing stagnant atmospheric conditions that are inherently included in observational data and models including meteorology, such as WRF-Chem.
Future work looking at the influence of temperature on ozone should include stagnant conditions to represent more realistic atmospheric conditions.
Any modelling work addressing this should also consider a range of \ce{NO_x} conditions as this strongly influenced the amount of ozone produced in our study.
