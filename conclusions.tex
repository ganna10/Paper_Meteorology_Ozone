In this study, we have simulated idealised urban conditions with a box model over a range of temperatures and \ce{NO_x} conditions using a temperature-independent and temperature-dependent source of isoprene emissions.
These simulations were repeated using reduced chemical mechanism schemes (CRIv2, MOZART-4, CB05 and RADM2) typically used in 3-D models and compared to the near-explicit MCMv3.2 chemical mechanism.
Each chemical mechanism produced a non-linear relationship of ozone with temperature and \ce{NO_x} with the most ozone produced under high emissions of \ce{NO_x}.
Conversely, lower \ce{NO_x} levels led to a minimal increase of ozone with temperature regardless of the source of isoprene.
Thus future air quality under the influence of higher temperatures due to climate change would benefit from reducing \ce{NO_x} emissions.

Allocating the ozone from each box model simulation to separate \ce{NO_x} regimes indicated that faster chemistry with temperature is responsible for a greater absolute increase in ozone than increased isoprene emissions.
The larger increase of ozone with temperature due to faster chemistry was reproduced by each chemical mechanism and in each \ce{NO_x} regime.

Production budgets of ozone were normalised by the total oxidation rate of emitted VOC and allocated to the major contributing processes: peroxy nitrate decomposition, reaction of NO with \ce{HO2}, alkyl and acyl peroxy radicals, inorganic reactions and any other organic reactions.
The total normalised ozone production budget in each \ce{NO_x} regime was larger in the MCMv3.2 than any reduced chemical mechanism.

The increase in thermal decomposition of peroxy nitrates with temperature has the largest contribution to ozone production with chemical mechanism and each \ce{NO_x} regime.
The contribution of peroxy nitrates is larger in MCMv3.2 due to the inclusion of methylperoxy nitrate chemistry that is not included in any other chemical mechanism used in this study.
Including methylperoxy nitrate chemistry in reduced chemical mechanisms would increase the number of molecules of ozone produced per molecule of emitted VOC oxidised.

The slope (m$_{\text{O3-T}}$) of the linear fit of ozone-temperature values from observational data (ERA-Interim) over Europe was twice as high as the closest slope using the box model (temperature-dependent emissions of isoprene and high emissions of \ce{NO_x}).
Compared to WRF-Chem output using MOZART-4 and RADM2 chemistry, the ozone-temperature values from the box model are again less-sensitive to temperature than the WRF-Chem output.
The box model does not represent stagnant atmospheric conditions that are represented by observational values and simulated by 3-D models that include meteorology, such as WRF-Chem.
Future work looking at the influence of temperature on ozone should include stagnant conditions to represent more realistic atmospheric conditions.
Any modelling work addressing this should also consider a range of \ce{NO_x} conditions as this influences the amount of ozone produced.
