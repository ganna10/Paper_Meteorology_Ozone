In this study, we determined the effects of temperature on ozone production using a box model over a range of temperatures and \ce{NO_x} conditions using a temperature-independent and temperature-dependent source of isoprene emissions.
These simulations were repeated using reduced chemical mechanism schemes (CRIv2, MOZART-4, CB05 and RADM2) typically used in 3D models and compared to the near-explicit MCMv3.2 chemical mechanism.
Each chemical mechanism produced a non-linear relationship of ozone with temperature and \ce{NO_x} with the most ozone produced at high temperatures and high emissions of \ce{NO_x}.
Conversely, lower \ce{NO_x} levels led to a minimal increase of ozone with temperature.
Thus air quality in a future with higher temperatures predicted with climate change would benefit from dramatically reducing \ce{NO_x} emissions.

Faster chemistry at higher temperatures was responsible for a greater absolute increase in ozone than increased isoprene emissions.
Faster thermal decomposition of peroxy nitrates at higher temperatures contributed the most to ozone production with each chemical mechanism and all \ce{NO_x} conditions.
The contribution of peroxy nitrates using reduced chemical mechanisms was larger than in MCMv3.2 due to the inclusion of methylperoxy nitrate (\ce{CH3O2NO2}) chemistry that is not included in any other chemical mechanism used in this study.
Including methylperoxy nitrate chemistry in reduced chemical mechanisms would minimise the differences in the production of ozone from reduced chemical mechanisms to the MCMv3.2 at higher temperatures.

The rate of change of ozone with temperature using observational data (ERA-Interim) over Europe was twice as high as when using the box model.
This was due to the box model not representing stagnant atmospheric conditions that are inherently included in observational data and models including meteorology, such as WRF-Chem.
Future work looking at the influence of temperature on ozone should include stagnant conditions to represent more realistic atmospheric conditions.
Any modelling work addressing this should also consider a range of \ce{NO_x} conditions as this strongly influenced the amount of ozone produced in our study.
