In this study, we determined the effects of temperature on ozone production using a box model over a range of temperatures and \ce{NO_x} conditions with a temperature-independent and temperature-dependent source of isoprene emissions.
These simulations were repeated using reduced chemical mechanism schemes (CRIv2, MOZART-4, CB05 and RADM2) typically used in 3D models and compared to the near-explicit MCMv3.2 chemical mechanism.

Each chemical mechanism produced a non-linear relationship of ozone with temperature and \ce{NO_x} with the most ozone produced at high temperatures and moderate emissions of \ce{NO_x}.
Conversely, lower \ce{NO_x} levels led to a minimal increase of ozone with temperature.
Thus air quality in a future with higher temperatures would benefit from dramatical reductions in \ce{NO_x} emissions.

Faster reaction rates at higher temperatures were responsible for a greater absolute increase in ozone than increased isoprene emissions.
In our simulations, ozone production was controlled by the increased rate of VOC oxidation with temperature.
The net production of \ce{O_x} per oxidised molecule of emitted VOC with temperature was regulated by the decrease of ozone deposition per molecule of emitted VOC oxidised.

The rate of increase of ozone with temperature using observational data over Europe was twice as high as the rate of increase of ozone with temperature when using the box model.
This was consistent with our box model setup not representing stagnant atmospheric conditions that are inherently included in observational data and models including meteorology, such as WRF-Chem.
Model simulations without mixing determined the sensitivity of the ozone-temperature relationship to mixing and in these model runs the rate of increase of ozone with temperature was faster than the simulations including mixing.
The simulations without mixing with VOC-and-\ce{NO_x} sensitive chemistry led to very similar rates of increase of ozone with temperature to the observational and WRF-Chem data.
The sensitivity of ozone production to temperature and \ce{NO_x} conditions should be considered in future modelling work especially when looking at scenarios for reducing the burden of ozone pollution in the future.
