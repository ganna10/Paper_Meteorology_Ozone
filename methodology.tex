\subsection{VOC Tagging Approach}
\begin{itemize}
    \item boxmodel set-up, initial conditions, VOC emissions same as in \citet{Coates:2015}
    \item only difference is the tagging approach used; \citet{Coates:2015} determine effects of VOC on O3 by inferring the effects of VOC on Ox production, hence this is looking at the effects on O3 indirectly
    \item the tagging approach of \citet{Emmons:2012}, which looks at NOx-tagging has been adapted to VOC tagging by Shuai and Tim and now the effects of VOC on O3 mixing ratios can be directly compared
    \item Tagging the VOC degradation of MOZART-4 was achieved using the same technique as S\&T but using the KPP version of MOZART-4, including the modifications to MOZART-4 outlined in \citet{Coates:2015} such as using MCM~v3.2 inorganic chemistry, using only reactions relevant to tropospheric processes.
    \item From the gas-phase reactions in the KPP version, we have the full set of non-tagged reactions -- the ``real'' chemistry -- and appended this with reactions where the degradation reactions of a VOC are tagged for the VOC in order to trace the effects of the VOC degradation on O3 and Ox production.
    \item The tagging approach requires extracting the set of reactions where reactions of the VOC products with members of the Ox family are also included.
    \item We define the Ox family to include O3, O, O1D, NO2, NO3, N2O5, HNO3, ISOPNO3, ONIT, ONITR, HO2, HO2NO2.
    \item The set of tagged chemistry which runs parallel to the ``real'' chemistry was then added to the KPP version of MOZART-4 chemical mechanism described above.
    \item This chemistry was then implemented in the boxmodel and then tested to determine that the tagged chemistry does not influence the ``real'' chemistry by comparing the mixing ratios of tagged species to their non-tagged counterparts. If the chemistry is correctly set-up then there should be no difference.
    \item The next testing stage was that this new chemistry set-up gives the same results as the previous chemistry (i.e. the results in \citet{Coates:2015}) by comparing the mixing ratios of the ``real'' chemistry in the new set-up to that in \citet{Coates:2015} and again there should be no differences as the boxmodels are set-up in exactly the same way.
\end{itemize}
