%\subsection{VOC Tagging Approach} \label{ss:O3_tagging}
%\begin{itemize}
%    \item boxmodel set-up, initial conditions, VOC emissions same as in \citet{Coates:2015}
%    \item only difference is the tagging approach used; \citet{Coates:2015} determine effects of VOC on O3 by inferring the effects of VOC on Ox production, hence this is looking at the effects on O3 indirectly
%    \item the tagging approach of \citet{Emmons:2012}, which looks at NOx-tagging has been adapted to VOC tagging by Shuai and Tim and now the effects of VOC on O3 mixing ratios can be directly compared
%    \item Tagging the VOC degradation of MOZART-4 was achieved using the same technique as S\&T but using the KPP version of MOZART-4, including the modifications to MOZART-4 outlined in \citet{Coates:2015} such as using MCM~v3.2 inorganic chemistry, using only reactions relevant to tropospheric processes.
%    \item From the gas-phase reactions in the KPP version, we have the full set of non-tagged reactions -- the ``real'' chemistry -- and appended this with reactions where the degradation reactions of a VOC are tagged for the VOC in order to trace the effects of the VOC degradation on O3 and Ox production.
%    \item The tagging approach requires extracting the set of reactions where reactions of the VOC products with members of the Ox family are also included.
%    \item We define the Ox family to include O3, O, O1D, NO2, NO3, N2O5, HNO3, ISOPNO3, ONIT, ONITR, HO2, HO2NO2.
%    \item The set of tagged chemistry which runs parallel to the ``real'' chemistry was then added to the KPP version of MOZART-4 chemical mechanism described above.
%    \item This chemistry was then implemented in the boxmodel and then tested to determine that the tagged chemistry does not influence the ``real'' chemistry by comparing the mixing ratios of tagged species to their non-tagged counterparts. If the chemistry is correctly set-up then there should be no difference.
%    \item The next testing stage was that this new chemistry set-up gives the same results as the previous chemistry (i.e. the results in \citet{Coates:2015}) by comparing the mixing ratios of the ``real'' chemistry in the new set-up to that in \citet{Coates:2015} and again there should be no differences as the boxmodels are set-up in exactly the same way.
%    \item The final boxmodel setup was used as the base case for each of the subsequent modelling scenarios, and simulates NOx conditions that produce maximum ozone for each VOC.
%\end{itemize}

\subsection{Model Setup} \label{ss:model_setup}
\begin{itemize}
    \item MECCA box model as described in \citet{Coates:2015} to broadly simulate the benelux (Belgium, Netherlands and Luxembourg) region. Solar zenith angle of $51^{\circ}$N was used to determine photolysis rates through a parameterisation and the SZA chosen is broadly representative of the central benelux region.
    \item MECCA box model has been updated to include vertical mixing with the free troposphere and accordingly includes a diurnal cycle for the PBL height. These amendments are discussed further in Sect.~\ref{ss:vertical_mixing}.
    \item All simulations performed using the Master Chemical Mechanism, MCM~v3.2, \citep{MCM_Site} and also repeated using MOZART-4 \citep{Emmons:2010}. \citet{Coates:2015} describes the implementation of both MCM~v3.2 and MOZART-4 for use with KPP within MECCA.
    \item NOx and other parameters were varied systematically to analyse the effects on ozone mixing ratios over different NOx gradients and hence different atmospheric conditions.
    \item VOC emissions constant until noon of first day, to simulate a plume of emitted VOC.
    \item Simulations start at 06:00 using spring ecquinoctical conditions and the simulations ended after two days.
\end{itemize}

\subsection{VOC Emissions} \label{ss:VOC_emissions}
\begin{itemize}
    \item Anthropogenic emissions from the Benelux regions were determined using the TNO-MACC\_III emission inventory; the TNO-MACC\_III inventory is the latest update of the TNO-MACC\_II inventory and was created using the same methodology as \citep{Kuenen:2014} and based upon improvements to the existing emission inventory during the AQMEII~2 exercies described in \citet{Pouliot:2015}. 
    \item Temperature independent emissions of the biogenic VOC isoprene and monoterpenes, were calculated as a fraction of the total anthropogenic VOC emissions from each country in the Benelux region, this data was obtained from the supplementary data available from the EMEP (European Monitoring and Evaluation Programme) model \citep{Simpson:2012}.
    \item These emissions are included as total emissions from SNAP (Selected Nomenclature for Air Pollution) source categories and were assigned to chemical groupings based on the country specific profiles for Belgium, the Netherlands and Luxembourg provided in \citet{Builtjes:2002}. \todo{correct citation?}
    \item The MCM~v3.2 initial species were determined using the country specific profiles for each SNAP source category and where appropriate information of individual chemical species that can be represented by MCM~v3.2 were determined using the speciations of \citet{Passant:2002}. %%include examples from Solvents paper
    \item After calculating the MCM~v3.2 initial VOC and respective emissions were assigned to the respective MOZART-4 species and the emissions in MOZART-4 were weighted by the carbon numbers of the MCM~v3.2 species and the emitted MOZART-4 species.
\end{itemize}

\subsection{Vertical Mixing with Diurnal Boundary Layer Height} \label{ss:vertical_mixing}
%most likely include in supplement
\begin{itemize}
    \item The base boxmodel (Sect. \ref{ss:model_setup}) includes a constant boundary layer height of $1$ km and no interactions (mixing) with the free troposphere.
    \item A parameterisation of the diurnal profile of the planetary boundary layer (PBL) height over Los Angeles was provided by Boris Bonn based on data from the CARES field campaign \citep{CARES:2008} \todo{check if reference correctly}.
    \item The PBL height was calculated at every time point for the model run and then read into the boxmodel at each time point \todo{include plot}.
    \item The concentrations of the chemical species within the PBL are diluted due to the larger mixing volume when the PBL height increases at the beginning of the day, also the increasing PBL height induces mixing of chemical species from the free troposphere with those chemical species within the PBL i.e. vertical mixing. When the PBL height collapses during night giving the stable nocturnal boundary layer, this traps the chemical species into a smaller volume thus increasing the concentrations of the chemical species.
    \item This vertical mixing scheme was implemented into the boxmodel using the same approach of \citet{Lourens:2012}.
    \item The mixing ratios of O3, CO and CH4 in the free troposphere were respectively set to $50$ ppbv, $116$ ppbv and $1.8$ ppmv. These condions were taken from the MATCH-MPIC chemical weather forecast model on the 27th March (the start date of the simulations). The model results (\url{http://cwf.iass-potsdam.de/}) at the 700 hPa height were chosen and the daily average was used as input into the boxmodel. \todo{check about reference and if this is appropriate}
    \item Tagged free troposphere species were also included in the boxmodel to determine effect of free troposphere species on surface ozone levels.
\end{itemize}
