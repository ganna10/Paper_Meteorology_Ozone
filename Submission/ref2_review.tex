\documentclass{article}

\usepackage{setspace}
\setstretch{1.5}
\usepackage[a4paper, margin=25mm]{geometry}
\usepackage{microtype}
\usepackage[round]{natbib}
\setlength{\bibhang}{0pt}
\usepackage{hyperref}

\begin{document}

We would like to thank the reviewer for the review that will enable us to improve our manuscript; our responses to the review points and comments are found below.

\section*{General Comments}

\textbf{Review Point 1:}  Generally the paper is very short. I realise that keeping things brief and to the point is sometimes a good thing and can help the reader concentrate on the salient points, however I would suggest in this case that some of the supplementary material be moved to the main text. In particular I think the model setup section would benefit from having more description in the main text rather than most of it being in the supplementary. This is important information for the paper and in this case I believe it would assist the reader to expand the model description.

\textbf{Author Response:} We thank the reviewer for required additional information about the model description, hence we shall incorporate the details of the Section in the supplement titled 'Vertical Mixing with Diurnal Boundary Layer Height' in to Section 2.1 of the article.
The first paragraph shall include: \textit{The vertical mixing scheme was based on the approach of \citet{Lourens:2016} with the model using the mean mixing layer height from the BAERLIN campaign over Berlin, Germany \citep{Bonn:2016}.}

Furthermore, the second paragraph shall include the following information:
\textit{These conditions were taken from the MATCH-MPIC chemical weather forecast model on the 21st March (the start date of the simulations). 
The model results (\url{http://cwf.iass-potsdam.de/}) at the 700 hPa height were chosen and the daily average was used as input into the boxmodel.  
}

\section*{Specific Comments:}
\textbf{Specific Comments 1:}  In section 2.1 (page 3 line 30 – page 4 line 4), several statements are made about the setup of the model that would benefit from expansion. The authors state that isoprene emissions from vegetation are the most important BVOC emissions on a global scale, however if the study was to be used for mechanisms in regional as well as global models, then could other BVOCs and other isoprene sources become important? For example in moderate to high NOx conditions of large cities could anthropogenic isoprene be important? And could monoterpene emissions (which have a potentially large effect on O3 chemistry due to their reaction rate with OH and O3 itself) also be significant? In general this seems to be a big statement to make without further discussion. The authors also state (page 4 line 3) that AVOC emissions can be effected by increased temperature due to increase evaporation but then have no further discussion as to how omitting this temperature dependence from the study may affect the results.

\textbf{Author Response:} We agree that the statements regarding isoprene emissions requires further discussion, accordingly we shall update the paper (details) with the following text.
\textit{
    jsjs
}

Furthermore, the statement relating to evaporative emissions of anthropogenic VOC (page 4 line 3) includes the following discussion.
\textit{
    text
}

\textbf{Specific Comments 2:} On page 4 line 30 it is described how isoprene emissions with varying temperature using MEGAN2.1 lead to different isoprene mixing ratios in the model, and this is then compared to isoprene measured at different temperatures during a campaign over Essen, Germany. This needs expanding. I presume MEGAN was run in the model for the particular area that the campaign took place over but this needs stating explicitly.  Could the authors check their model with other campaigns that have measure isoprene (of which there are numerous worldwide in the literature)?

\textbf{Author Response:}

\textbf{Specific Comments 3:} On page 5 line 30 a description is given that the increase in ozone due to chemistry is large than that due to increased emissions. The results are shown in figure 3 and table 2, however the paper would greatly benefit from a summary of the results in the text. 

\textbf{Author Response:}

\textbf{Sepcific Comments 4:} On page 7 line 16, there is a paragraph describing how faster reaction of VOCs with OH with increased temperature can increase ozone production. This is backed up by references to other studies that have seen this effect. Why have the authors not included the results of their study here? Could they include some description of which VOC + OH reactions are most dependant on temperature, which would assist readers in coming to a conclusion about which reactions and their temperature dependence should be included in any given model?

\textbf{Author Response:}

\textbf{Specific Comments 5:} In section 3.3, a description is given of how the box model simulations in this study compare to real-world observations and the output of various 3-D models. I must admit I am a bit confused what this section is trying to say. It seems that the result is that mixing in the box model is more important to ozone formation that the choice of mechanism (which is not surprising) and I am not quite sure how any useful comparison can be made between the different mechanisms in this study and a few real world and 3-d model studies. Maybe the authors could better explain what they are trying to achieve with this section. Would a better approach be to assess what mechanisms were used in the various studies they look at and then give some steer as to whether it is the temperature dependence of the chemistry or of the emissions that is the key driver in these different cases?

\textbf{Author Response:}

\section*{Minor Comments:}

\textbf{Minor Comments 1:} Page 1 line 22: Could more references be added here – especially
with respect to the many studies of the 2003 European heatwave ozone events?

\textbf{Author Response:} 

\textbf{Minor Comments 2:}  Page 3 line 13: What was ‘broadly representative of urban conditions of central Europe’ mean. Please be more specific with the conditions the model was run at.

\textbf{Author Response:}

\textbf{Minor Comments 3:} Page 3 line 27: The Stockwell 1990 reference seems very old. Has there been more recent advances in the knowledge of ozone production chemistry that might make this obsolete?

\textbf{Author Response:}

\textbf{Minor Comments 4:}  Page 8 line 25: The authors should consider showing the actual production and consumption budgets in the main text rather than the supplementary.

\textbf{Author Response:}


\begin{thebibliography}{2}

    \bibitem[{Lourens et~al.(2016)Lourens, Butler, Beukes, van Zyl, Fourie, and, Lawrence}]{Lourens:2016} Lourens,~A.~S.~M., Butler,~T.~M., Beukes,~J.~P., van Zyl,~P.~G., Fourie,~G.~D., and Lawrence,~M.~G.:Investigating atmospheric photochemistry in the Johennesburg-Pretoria megacity using a box model, South African Journal of Science, 112, 1/2, 2016.

    \bibitem[{Bonn et~al.(2016)Bonn, von Schneidemesser, Andrich, Quedenau, Gerwig, L\"udecke, Kura, Pietsch, Ehlers, Klemp, Kofahl, Nothard, Kerschbaumer, Junkermann, Grote, Pohl, Weber, Lode, Sch\"onberger, Churkina, Butler, and, Lawrence}]{Bonn:2016} Bonn,~B., von Schneidemesser,~E., Andrich,~D., Quedenau,~J., Gerwig,~H., L\"udecke,~A., Kura,~J., Pietsch,~A., Ehlers,~C., Klemp,~D., Kofahl,~C., Nothard,~R., Kerschbaumer,~A., Junkermann,~W., Grote,~R., Pohl,~T., Weber,~K., Lode,~B., Sch\"onberger,~P., Churkina,~G., Butler,~T.~M., and, Lawrence,~M.~G.: BAERLIN2014 - The influence of land surface types on and the horizontal heterogeneity of air pollutant levels in Berlin, Atmospheric Chemistry and Physics Discussions, 1--62, 2016.

\end{thebibliography}

\end{document}
