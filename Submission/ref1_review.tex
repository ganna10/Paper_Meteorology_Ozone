\documentclass{article}

\usepackage{setspace}
\setstretch{1.5}
\usepackage[a4paper, margin=25mm]{geometry}
\usepackage{microtype}
\usepackage[round]{natbib}
\setlength{\bibhang}{0pt}
\usepackage{hyperref}

\begin{document}

We would like to thank the reviewer for the review that will enable us to improve our manuscript; our responses to the review points and comments are found below.

\section*{General Comments}

\textbf{Review Point 1:}  Generally, the paper needs clearer focus and to supply more model details/results/discussion pertinent to this focus. While the stated goal is to assess the temperature dependence of ozone chemistry as a function of NOx, at the end, I am not exactly sure what new has been learned. Really, what I think is missing is just a dedicated discussion section, the `Results' read like a true results section, rather than combined results and discussion, so it is difficult for the reader to understand the significance of the calculations.

\textbf{Author Response:} 

\textbf{Review Point 2:} The reasons behind the ozone impacts from temperature-dependent reaction rates are unclear because it is not stated explicitly what this term includes. If I understand correctly, only temperature-dependent reaction rates, k(T), are being tested (page 4, lines 1–6). Generally, the model description does not give the reader sufficient information to understand what causes the changes reported in Table 2. For example, the text states the RONO2 formation is temperature dependent, at least in some mechanisms, but does this refer to the RONO2 branching ratio?

\textbf{Author Response:}

\textbf{Review Point 3:} It is interesting to see how the five different mechanisms capture these effects; however, there is little discussion of what is learned about the different mechanisms by testing them in this way. Can the last paragraph of Sect. 3.1 be expanded? Also, I have difficulty discerning differences between panels in Fig. 2. Is there a way to highlight key differences here? Regarding Fig. 3, something has been lost in translation - there are some floating numbers in the upper left corner of the top-left panel, but the different colors are not labeled, either in the text or caption. I deduce that green is CB05, purple is MCM3.2, blue is MOZART-4, and orange and yellow are each either RADM2 or CRIv2. I would be interested to read more, not just about what causes the differences between the curves, but also the implications for studying air quality and chemistry. Finally, the NOx regime distinction is derived from each individual model's H2O2:HNO3. Why not simply use the shape of simulated PO3 versus NOx. A missing piece of information is whether maximal O3 - at different temperatures - occurs at the same NOx level in each mechanism. Differences in the NOx level of maximal O3 for different mechanisms have consequences for air quality decision-making.

\textbf{Author Response:}

\textbf{Review Point 4:}  Provide a statement as to why the ozone production and consumption budget is informative for understanding the temperature dependence of ozone, i.e.  what is gained compared to thinking about production alone. Also, can an equation be provided for the production and consumption budgeting? This section is in need of discussion and summary. There are many panels in Fig. 4 and it is not obvious to me what the take-away point(s) are.

\textbf{Author Response:}

\textbf{Review Point 5:} Before the authors talk about mixing, the WRF-CHEM and box MOZART- 4 results should be compared directly and discussed. The importance of atmospheric mixing appears for the first time in Section 3.3, at which time, the paper states it is the most important term in mO3-T. At this stage in the manuscript, I am left wondering what is this paper actually about. How does Section 3.3 relate to the previous two sections?  A subsequent discussion would be helpful.

\textbf{Author Response:} 

\section*{Minor Comments:}
\textbf{Minor Comments 1:}  More information should be provided in the introduction. The three sentences in the paper’s first paragraph do not really follow logically. I am not familiar with the Otero paper and this single-sentence description does not stand on its own–temperature was shown to be a driver of which process?

\textbf{Author Response:} 

\textbf{Minor Comments 2:} Fig. 2: The ozone contours are labeled left to right: 5, 50, 55, 0, 5, 0, 5. The y-axis reads: 10, 10, 30, 50. On the x-axis, the 4 of 40 has been lost.

\textbf{Author Response:}

\textbf{Minor Comments 3:} Fig. 5: The majority of measured O3 data are found at lower temperatures, so fitting the calculated O3 with a straight line across the whole temperature range may not be representative.

\textbf{Author Response:}

\textbf{Minor Comments 4:} Fig. 5: y-axis reads 25, 50, 5, 100, 125; on the x-axis, the 4 of 40 has been lost.

\textbf{Author Response:}

\begin{thebibliography}{2}

    \bibitem[{Lourens et~al.(2016)Lourens, Butler, Beukes, van Zyl, Fourie, and, Lawrence}]{Lourens:2016} Lourens,~A.~S.~M., Butler,~T.~M., Beukes,~J.~P., van Zyl,~P.~G., Fourie,~G.~D., and Lawrence,~M.~G.:Investigating atmospheric photochemistry in the Johennesburg-Pretoria megacity using a box model, South African Journal of Science, 112, 1/2, 2016.

    \bibitem[{Bonn et~al.(2016)Bonn, von Schneidemesser, Andrich, Quedenau, Gerwig, L\"udecke, Kura, Pietsch, Ehlers, Klemp, Kofahl, Nothard, Kerschbaumer, Junkermann, Grote, Pohl, Weber, Lode, Sch\"onberger, Churkina, Butler, and, Lawrence}]{Bonn:2016} Bonn,~B., von Schneidemesser,~E., Andrich,~D., Quedenau,~J., Gerwig,~H., L\"udecke,~A., Kura,~J., Pietsch,~A., Ehlers,~C., Klemp,~D., Kofahl,~C., Nothard,~R., Kerschbaumer,~A., Junkermann,~W., Grote,~R., Pohl,~T., Weber,~K., Lode,~B., Sch\"onberger,~P., Churkina,~G., Butler,~T.~M., and, Lawrence,~M.~G.: BAERLIN2014 - The influence of land surface types on and the horizontal heterogeneity of air pollutant levels in Berlin, Atmospheric Chemistry and Physics Discussions, 1--62, 2016.

\end{thebibliography}

\end{document}
