Many observational studies have noted an almost linear increase of ozone levels with temperature.
The reasons for this increase are two-fold -- temperature-dependent emissions of ozone precursors, the most important being the increase in isoprene emissions from vegetation, and temperature-dependent chemistry leading to ozone production.
We look at how the relationship between ozone and temperature is represented in idealised simulations using a box model and repeated using different chemical mechanisms across different NOx gradients.
What is more important for the increase of ozone with temperature? Increased emissions of isoprene or the increase in the rates of chemical reactions? How does this change with NOx?
